\mode*

\setmonofont{Verdana}

\newtheorem{beispiel}{Beispiel}[section]
\newtheorem*{bemerkung}{Bemerkung}

% ===========================================================================
\section{Modulidentifkation}

\definecolor{mydarkgrey}{RGB}{96,96,96}
\definecolor{mygrey}{RGB}{224,224,224}

\rowcolors{1}{mygrey}{white}

\begin{center}
    \begin{tabular}[h]{|p{4cm}|p{12.5cm}|}
        \hline
        Modulnummer & 150 \\ \hline
        Titel & E-Business-Applikationen anpassen \\ \hline
        Kompetenz & E-Business-Applikationen gemäss Vorgabe und unter Beachtung der
        Sicherheitsvorschriften und technischer Rahmenbedingungen anpassen. \\ \hline
        \multirow{1}{*}{Handlungsziele} &
        Aufbau der Applikation, Transaktionskonzept, Applikationsumgebung
        und Rahmenbedingungen (Sicherheit, Performance, Verfügbarkeit, Transaktionsvolumen, usw.) erfassen. \\
        \cellcolor{white}  & Vorgabe analysieren, clientseitigen, serverseitigen und datenbankseiti-gen
        Änderungsbedarf formulieren. \\
        \cellcolor{white}  & Auswirkungen der Änderungen auf Sicherheit und Schutzwürdigkeit der
        Informationen bei allen beteiligten Komponenten wie Client, Webserver, Applikationsserver
        und Datenbankserver überprüfen und dokumentieren. \\
        \cellcolor{white}  & Änderungen inklusive Implementierung und Test (funktional und nicht-funktional)
        gemäss einem vordefinierten Änderungsprozess planen. \\
        \cellcolor{white}  & Änderungen realisieren, testen und dokumentieren. \\ \hline
        Kompetenzfeld & Web Engineering \\ \hline
        Objekt & Online-Shop, Ticketing, Wiki oder E-Learning Lösung. \\ \hline
        Niveau & 4 \\ \hline
    \end{tabular}
\end{center}

% ===========================================================================
\section{Einleitung und Rahmenbedingungen}

\subsection{Begriffsdefnitionen - Modul 150}

Mit dem Präfix 'E' versehen, geistern in der Informatik viele Begriffe herum.
Auch in der Modulidentifikation zu diesem Modul werden die Begriffe \emph{E-Business}
und \emph{E-Commerce} parallel verwendet. Je nach verwendeter Literatur, werden
zwischen diesen beiden Begriffen aber wesentliche Unterschiede gemacht.
In diesem Skript werden wir uns aus mühsamen Begriffsdefinitionen möglichst
heraus halten und einen Überblick über Techniken und Grundlagen geben,
die bei der Anpassung und beim Entwurf von Webapplikationen eine Rolle
spielen. Wenn hier im Skript der Begriff \emph{E-Business} verwendet wird,
bezeichnet er Applikationen, die mindestens die Komponenten Produktauswahl,
Zahlungsverkehr und Warentransport (elektronisch oder physikalisch) enthalten.

Soziale Plattformen, Foren und so weiter gehören nicht zu dieser
Applikationsgruppe. Die meisten hier betrachteten Techniken werden
auch bei diesen Anwendungen eingesetzt.

Sie sollten in diesem Modul einen Überblick über Themen bekommen, die im
Bereich \emph{E-Business} wichtig sind. Die können in 40 Lektionen zwar nicht
vollständig bearbeitet werden, aber eine Sensibilisierung auf mögliche
Probleme und der Erwerb von Grundkenntnissen mit zusätzlicher selbstständiger
Vertiefung sind sicher möglich.

\subsection{Projektmanagement}

Die Abwicklung von Webprojekten und die Realisierung von E-Business Lösungen
erfordert ein professionelles Projektmanagement. Welche Methoden dabei
eingesetzt werden, ist von den beteiligten Unternehmen und von der Anwendung
abhängig und nicht Bestandteil dieses Moduls. Lediglich der Begriff \emph{Change
Management} sei hier noch erwähnt, auf den wir später kurz eingehen werden.


\subsection{Partnerrollen bei Webapplikationen}

Auch wenn die Begriffsdefinitionen hier nicht im Vordergrund stehen, müssen
wir eine Reihe von Merkmalen kennen, da sie die Applikation und auch die
eingesetzte Technik stark beeinflussen. Wir müssen uns klar machen, welche
Unterschiede in Bezug auf die Geschäftspartner bei Webapplikationen bestehen
und in welchem Gebiet unsere Applikation eingesetzt wird. Das hat Konsequenzen
auf den Entwurf, die anzuwendenden Gesetze, die Technik und die Gestaltung
der Applikation.

Anmerkung bezüglich der Partnerrolle 'C': In der Literatur
und auch im Internet finden wir die Bezeichnungen \emph{customer} und \emph{consumer}.
In Verbindung mit der Rolle 'A' für \emph{administration} bedeutet es dann
\emph{citizen} also Bürger.

\subsubsection*{B2C - Business to Customer/Consumer}

Diese Geschäftsbeziehung fällt einem vermutlich als erste mögliche Beziehungsvariante
ein. Sie entspricht dem klassischen Ladengeschäft, das seine Waren an Endkunden
verkauft. In der elektronischen Variante ist dazu kein physikalischer Laden mehr
nötig, aber viele der gesetzlichen Bestimmungen zum Thema Konsumentenschutz, sind
mittlerweile auch im Internet-Handel gültig. Etwas komplizierter wird es durch die
Globalisierung via Internet. Wenn zwischen den Ländern des Verkäufers und des Käufers
Gesetze und Vereinbarungen bestehen, ist die Unsicherheit zwar limitiert, aber Zölle,
Einfuhrbeschränkungen auf bestimmte Güter, Versandmodalitäten und eventuell auch
sprachliche Schwierigkeiten sorgen für ausreichend Möglichkeiten einen Geschäftsvorgang
kompliziert werden zu lassen. Was die juristischen Anforderungen zum Thema
Konsumentenschutz betrifft, sind hier besonders das Rückgaberecht und die
Auszeichnungspflicht (Angabe des Warenpreises inklusive Mehrwertsteuer) zu nennen.
Diese Konstellation der Geschäftspartner entspricht auch meistens den von uns
untersuchten WebshopAnwendungen.

\subsubsection*{B2B - Business to Business}

Hierbei handelt es sich um die Beziehung zweier Geschäftspartner. Beide Partner
sind sich im Normalfall bekannt. Zumindest muss eine Überprüfung der Autorisierung
erfolgen, um sicher zu stellen, dass es sich bei beiden Partnern um Firmen handelt.
Vor unberechtigten Zugriffen durch Endkunden muss die Anwendung geschützt werden.
Bei Verkaufsgeschäften bestehen keine Rücknahmepflichten und die Preisauszeichnung
erfolgt ohne Angabe der Mehrwertsteuer. Ein privater Kunde hat andere Anforderungen
und Erwartungen an eine Applikation als ein Lieferant oder Zwischenhändler.
Zumeist wird diesen mehr Können in Bezug auf \emph{Usability} und Ziele der
Applikation zugemutet als es bei anonymen Endkunden möglich ist. So kann die
Schnittstelle durchaus einmal aus einem autorisierten
FTP\footnote{File Transfer Protocol}-Download einer Datenbank bestehen.

\subsubsection*{C2C - Customer to Customer}

Das wohl erfolgreichste Beispiel für diese Kundenbeziehung ist EBay\textsuperscript{TM}.
Ursprünglich ging es hier um einen Gebrauchtwarenhandel, der zu Beginn
ausschliesslich zwischen Privatpersonen durchgeführt wurde. Aktuell tummeln
sich aber auch jede Menge Händler auf der Plattform und bieten Neuwaren an.
Allerdings habe diese die Verpflichtung sich klar als solche zu kennzeichnen
und die Bedingungen entsprechen dann einer B2C-Beziehung.

Findige (und zumeist auch windige) Juristenbüros haben diesen Umstand
ausgenutzt und untersuchen die Anbieter nach ihren Artikel- und Angebotsprofilen.
Wer eine alte CD-Sammlung auflöst, verkauft verschiedene CDs einmal und selten
vielleicht zweimal. Wer aber die gleiche CD mehrfach verkauft muss sich eine
gute Argumentation zurechtlegen, warum er sich nicht als Händler ausgibt
und muss entsprechend mit Abmahnungen rechnen (siehe auch
Kapitel~\ref{subsec:juristische-grundlagen} auf Seite~\pageref{subsec:juristische-grundlagen}).

\subsubsection*{A2C - Administration to A, B, C}

Es gibt eine ganze Reihe von eBegriffen, die im Bereich von Behördenapplikationen
verwendet werden. eGovernment, eAdministration, eJustice und eDemocracy sind
einige davon. Interessant ist die derzeitige Diskussion in den Diskussionsforen
der deutschen Wikipedia (zum Beispiel beim Begriff E-Government). Es wird
wohl noch einige Zeit verstreichen, bis sich feste Begriffe mit eindeutigen
Definitionen durchgesetzt haben. Generell geht es aber um Applikationen,
die Bürgern, Firmen oder anderen Behörden angeboten werden, um den Kontakt
auf elektronischem Weg zu ermöglichen. Das können stark sicherheitsrelevante
Anwendungen wie Steuerabrechnungen, E-Voting (wählen und abstimmen über das
Internet) und Ausschreibungen sein, aber auch einfache Applikationen, die den
Gang zur Verwaltung ersparen.

Ein Beispiel für eine \emph{Administration to Business}-Applikation finden Sie
auf der Website des
\href{https://www.blw.admin.ch/blw/de/home.html}{Bundesamts für Landwirtschaft BLW}.
Dort werden Einfuhrkontingente von landwirtschaftlichen Erzeugnissen wie Fleisch,
Wurstwaren, Zuchtrinder, Kartoffelprodukte periodisch versteigert. Allerdings
richtet sich die Web-Applikation eVersteigerung nicht an Gelegenheitsnutzer,
sondern erfordert eine vorherige Anmeldung und die Installation eines Zertifikates.
Anschliessend können Gebote für die ausgeschriebenen Importprodukte abgegeben
werden. Für die Bieter ist das nicht immer transparent und führt gelegentlich
zu Aufruhr, wie im Bericht ``Import von Filets ist eine Lotterie''
\footnote{Bericht im Tages-Anzeiger vom Montag, 08. Februar 2010} nachzulesen ist.

\subsubsection*{B2E - Business to Employee}

Anwendungen, die sich an die Angestellten einer Firma wenden, unterliegen
ebenfalls Bedingungen zum Datenschutz und auch zur Datensicherheit.
Besonders der Zugang von unautorisierten Personen ist hier ein Thema, denn
sonst wäre es ja keine reine B2E-Applikation mehr. Diese Applikationen
laufen meistens auf dem Intranet, da sie am Arbeitsplatz benötigt werden,
sie können aber teilweise auch via Internet sinnvoll sein, wie zum Beispiel
ein spezieller Webshop, der sich nur an Mitarbeiter richtet. Diese Kategorie
ist in der Übersicht nicht vertreten.

\rowcolors{2}{mygrey}{white}
\begin{table}
    \begin{center}
        \begin{tabular}[ht]{|p{3.5cm}|p{4cm}|p{4cm}|p{4cm}|}
            \hline
            \cellcolor{orange!25} &
            \cellcolor{orange!25}\textbf{Administration} &
            \cellcolor{orange!25}\textbf{Business} &
            \cellcolor{orange!25}\textbf{Consumer} \\ \hline
            \cellcolor{orange!25}\textbf{Administration} & \textbf{A2A} & \textbf{A2B} & \textbf{A2C}      \\
            \cellcolor{orange!25}                  & Applikation zwischen Behörden (z.B.~Interpol)
            & Applikationen für öffentliche Ausschreibungen &      \\ \hline
            \cellcolor{orange!25}\textbf{Business} & \textbf{B2A} & \textbf{B2B} & \textbf{B2C}  \\
            \cellcolor{orange!25}                  & & & Webshops; Anbieter sind gewerbsmässige Unternehmen  \\ \hline
            \cellcolor{orange!25}\textbf{Consumer} & \textbf{C2A} & \textbf{C2B} & \textbf{C2C}  \\
            \cellcolor{orange!25}                  & & Angebotssuche potentielle Kunden schreiben ihre Projekte aus, Firmen bieten an
            & Kleinanzeigen, Flohmärkte, alle privaten Handelsmöglichkeiten \\ \hline
        \end{tabular}
    \end{center}
    \caption{Anwendungsbereiche des eBusiness}
\end{table}


\subsection{Juristische Grundlagen}
\label{subsec:juristische-grundlagen}

Aufgrund der vermeintlichen (aber definitiv nicht realen) Anonymität im
Internet und der automatischen Globalisierung von Webauftritten, wird das
Internet von vielen als rechtsfreier Raum angesehen. Das kann aber
spätestens dann zu einem bösen Erwachen führen, wenn man eine Abmahnung
oder gar eine Klage erhält. Zumindest mit den wesentlichen rechtlichen
Begriffen und Gepflogenheiten muss man sich als Webentwickler auskennen.
Dabei ist es gleichgültig, ob es sich bei den erstellten Auftritten um
professionelle oder um private Seiten handelt.

\subsubsection*{Abmahnungen}

Bei Abmahnungen handelt es sich ursprünglich um Verfahrensvereinfachungen im
Wettbewerbsrecht. Wer einen Verstoss im Wettbewerbsrecht feststellt, war damit
nicht mehr gezwungen auf dem gerichtlichen Weg vorzugehen, sondern konnte
aussergerichtlich direkt mit der anderen Partei Kontakt aufnehmen und den
Fall lösen. Erst wenn sich die Partner nicht einigen können werden die
Gerichte damit belastet. Bei offensichtlichen Verstössen und einsichtigen
Konfliktparteien, können damit nicht nur die Gerichte entlastet werden, sondern
auch die Folgekosten für die widerrechtlich handelnde Partei stark reduziert
werden.

Soweit die grundsätzlich positive Idee dieses Rechtsmittels.

Leider wird dieses Instrument auch stark missbraucht. Es hat sich als ein
einträgliches Geschäft erwiesen, im Internet nach kleinen Verstössen zu
fahnden und die Urheber einfach abzumahnen, in der Hoffnung, dass diese---bei
relativ geringen Abmahngebühren---den Weg des geringsten Widerstandes nehmen
und bezahlen. Ein grosser Teil der Abmahngründe ist dabei juristisch nicht
haltbar. Natürlich hat sich auch eine Gegenbewegung gebildet, die bei Bedarf
Hilfe anbietet. Zum Beispiel \href{http://www.abmahnwelle.de}{http://www.abmahnwelle.de}.

\subsubsection*{Impressum}

Ein Impressum ist die Angabe von verantwortlichen Personen und deren Kontaktdaten.
Den Ursprung hat das Impressum bei Druckereierzeugnissen wie zum Beispiel
Zeitungen, Flugblättern und Werbeprospekten. Dort muss nicht nur der verantwortliche
Redakteur genannt werden, sondern auch die Adresse des Medienunternehmens und
eventuelle Beteiligungen, um mehr Transparenz in Bezug auf wirtschaftliche
Verflechtungen zu erhalten. In der Schweiz gilt die Impressumspflicht generell
für Medienhäuser und greift damit lange nicht so weit wie das EU-Recht.

Kompliziert werden internationale Kombinationen: eine Schweizer Internetfirma
hostet und entwirft den Auftritt für ein EU-Unternehmen und so weiter.
Neben diesen juristischen Finessen ist es bei eBusiness-Applikationen aus
Gründen der Transparenz und Offenheit wichtig, den Kunden nicht mit anonymen
Partnern arbeiten zu lassen. Ein Webauftritt, der den Betreiber im Dunkeln
lässt, wird kaum als seriös eingestuft und wird Mühe haben, zu den Kunden
eine bindende Beziehung herzustellen. Die meisten potentiellen Kunden werden
sich sträuben Informationen wie Adresse oder gar die Nummer der Kreditkarte
bekannt zu geben.

\subsubsection*{Disclaimer}

Die juristische Relevanz von Disclaimern wird in der Praxis zumeist überschätzt.
Grossfirmen haben gelegentlich am Ende ihrer Mails einen Hinweis auf die
Vertraulichkeit des Inhalts und dass bei fehlgeleiteten Mails deren Inhalt
nicht weitergegeben werden darf. Es ist halt ein Versuch, aber auch nicht
mehr. Denn juristisch entsprechen diese Hinweise Geschäftsbedingungen.
Und diese sind vor einer Geschäftshandlung bekannt zugeben und zu akzeptieren.
Korrekt, aber praktisch nicht machbar wäre es, zuerst den Disclaimer zu
senden und---falls der Empfänger die Bedingungen akzeptiert---erst anschliessend
den Inhalt.

Ebenso unwirksam sind die meisten Disclaimer auf Webseiten. Es ist---auch vor
Gericht---nicht glaubwürdig auf andere Webseiten zu verweisen und sich gleichzeitig
von diesen zu distanzieren. Wer in seriöser Absicht auf eine andere Webseite
verlinkt, sollte deren Inhalt auch kennen. Wenn der Inhalt nach dem Einbau des
Links verändert und eventuell juristisch bedenklich geworden ist, wird einem
das kaum als Vergehen angerechnet werden. Allerdings muss man nach einem
entsprechenden Hinweis darauf reagieren und seinen Eintrag anpassen.

\subsubsection*{Urheberrechtsschutz}

Ein juristisches Thema, das ebenfalls im Kapitel über eDistribution erwähnt
wird, ist das Urheberrecht. Es bezieht sich generell auf alle Medien. Hier
sei besonders auf die Situation im Bildbereich hingewiesen. Für die
Gestaltung der Produktkataloge werden zur Illustration oft Produktfotos
eingesetzt. Selbst wenn Sie diese nur von der Herstellerseite herunterladen
und verwenden, verstossen Sie gegen das Urheberrecht. Ausser die Fotos
sind explizit zur freien Verwendung deklariert. Selbst wer Fotos kauft,
muss sich über die Rechte seines Verkäufers informieren. Wenn später der
Fotograf seine Rechte geltend macht, kann die Verantwortung zwar an den
Verkäufer weitergegeben werden, nur hat man Pech, falls dieser Konkurs
gegangen ist oder nicht mehr auffindbar ist. Daher verwenden viele Firmen
für Ihre Webauftritte ausschliesslich Fotos, deren Urheberrecht sie selber
haben. Die also durch einen Fotografen speziell für sie erstellt werden.

% ===========================================================================
\section{Bestandteile des eBusiness}

\subsection{Übersicht der Wertschöpfungskette}

\subsection{Elemente der Wertschöpfungskette}

\subsubsection*{eProducts \& eServices}

Für den Anbieter von Waren ist die Präsentation seiner Produkte ein
zentraler Punkt. Die Darstellung, Beschreibung mit Attributen und
Fotos und Gliederung ermöglicht es den potentiellen Kunden ein
gewünschtes Produkt zu finden und bestenfalls auch zu bestellen. Für
diesen Zweck muss der Produktkatalog ständig auf dem aktuellen Stand
gehalten werden. Ein Aufwand, der bei grossen Systemen, die mehrere
hunderttausend Artikel enthalten können, nicht zu unterschätzen ist.
Bei komfortableren Systemen gibt es Möglichkeiten die Produkte hierarchisch
in Warengruppen zu gliedern. Damit können einige Produktattribute eventuell
schon auf der höheren Hierarchiestufe beschrieben werden und müssen
nicht redundant erstellt und gewartet werden.

Je nach Produkt besteht für den Kunden auch die Option seine Artikel
selber zu konfigurieren (zum Beispiel Dell: Konfiguration beziehungsweise
Modifikation von Ausstattungsvorschlägen). Dann benötigen einzelne
Komponenten noch weitere technische Angaben (Konfigurationsregeln), um
technisch falsche oder sinnlose Kombinationen zu verhindern. Der Aufbau
eines Konfigurationssystems ist natürlich aufwendiger und es hat auch
Auswirkungen auf die Produktion und Distribution der Produkte. Sie können
unter Umständen erst nach der Bestellung hergestellt werden, wenn es
sich bei der Konfiguration nicht nur um eine sehr geringe Anzahl an
Variationsmöglichkeiten handelt.

\subsubsection*{eProcurement}

Unter \emph{procurement} wird der Beschaffungsprozess in einem Unternehmen
verstanden. Mit dem E-Zusatz versehen, handelt es sich dann um die durch
spezielle Applikationen unterstützte elektronische Variante. Es
werden generell 3 Varianten unterschieden:

\begin{itemize}
    \item\textbf{Sell-Side}: Das System ist auf Verkäuferseite installiert.
    Der Aufwand mehrere Lieferantensysteme zu bedienen liegt beim Käufer.
    \item\textbf{Buy-Side}: Das System ist auf der Käuferseite installiert.
    Mit einem System können mehrere Lieferanten behandelt werden.
    \item\textbf{Marktplatz}: Plattform mit mehreren Anbietern und Nachfragern.
    Der Betreiber ist in der Regel eine unabhängige Drittfirma .
\end{itemize}

\subsubsection*{eMarketing}

Im Marketing gibt es die bekannte AIDA-Formel. Die Abkürzung steht für
\emph{Attention}, \emph{Interest}, \emph{Desire} und \emph{Action}.
So wie viele Erkenntnisse des klassischen Marketings ist auch diese
Formel im eMarketing gültig. Natürlich gibt es zusätzlich jede Menge
systemspezifische Merkmale. Die Einteilung der Kunden in Gruppen,
die dann vom Marketing separat betrachtet und mit Massnahmen gesegnet
werden können ist etwas anders als im klassischen Marketing. Wir
unterscheiden im eMarketing folgende Gruppen:

\begin{itemize}
    \item Online Surfer
    \item Online Consumer
    \item Online Prosumer \\
    (Mischung aus Produzent und Consumer---trägt zur Wertschöpfungskette
    positiv bei und ist gleichzeitig auch Konsument)
    \item Online Buyer und
    \item Online Key Customer
\end{itemize}


Ein bei Webshops häufig vernachlässigtes Gebiet ist die Verkaufs-Psychologie.
Wenn wir uns in Warenhäusern oder Supermärkten bewegen ist unsere gesamte
Umgebung nach psychologischen Kenntnissen gestaltet. Das ist nicht nur
die Akustik (also zumeist eine Hintergrundmusik), sondern kann auch
olfaktorisch (Geruchsstoffe) und optisch über die Beleuchtung erreicht
werden. Es ist zum Beispiel bekannt, dass die Gestaltung des Bodens
(hart oder weich) einen starken Einfluss auf unsere Gehgeschwindigkeit hat
und wird entsprechend eingesetzt. Eine unvollständige und etwas willkürliche
Aufzählung finden Sie zum Beispiel unter
\href{http://www.orbit9.de/wissen/verkaufspsychologie.php}
{http://www.orbit9.de/wissen/verkaufspsychologie.php}.
Mittlerweile gibt es aber auch im online-Bereich schon mehr Erfahrung,
die auch in professionellen Systemen eingesetzt wird. Diese Erkenntnisse
werden aber nicht sehr freizügig verbreitet. Wie sehr solche psychologischen
Tricks dann trotzdem auf Personen wirken, die sie kennen ist dann noche
eine weitere Diskussion. Das Wissen, dass zuviel Alkoholkonsum Kopfschmerzen
verursacht verhindert den Konsum ja auch nicht vollständig.

Weitere wichtige Punkte im eMarketing sind cross-selling und up-selling,
die noch im weiteren Verlauf des Skriptes behandelt werden.

\subsubsection*{eContracting}

\begin{itemize}
    \item\textbf{elektronischer Verhandlungsprozess} \\
    Softwaresysteme, die den elektronischen Verhandlungsprozess
    unterstützen, müssen die Dokumente und Unterlagen des gesamten
    Geschäftsprozesses verwalten und archivieren. Dazu gehören sämtliche
    Vereinbarungen und Vertragsabschlüsse, aber auch die digitalen
    Signaturen und das Controlling nach dem Vertragsabschluss.
    \item\textbf{digitale Signatur} \\
    Um rechtsgültige Vertragsabschlüsse im Internet zu erhalten,
    müssen sich die Geschäftspartner eindeutig identifizieren.
    Dies geschieht mittels digitaler Signaturen
    (siehe auch Kapitel~\ref{subsec:digitale-signatur} auf Seite~\ref{subsec:digitale-signatur}).
    Dokumente, die mit gültigen Signaturen versehen sind, sind rechtswirksam.
\end{itemize}

\subsubsection*{eDistribution}

Die Auswahl eines Distributionssystems sorgt für die Verteilung der Produkte,
Waren oder Dienstleistungen an die Kunden. Die Definition des Distributionskanals legt
fest, ob ein Produkt direkt oder indirekt abgesetzt wir. Beim indirekten Absatz
findet der Vertrieb nicht direkt zwischen dem Hersteller und dem Endkunden statt,
sondern es gibt eine oder mehrere Zwischenhändlerstufen. Bei digitalen Produkten
nennen sich die Zwischenhändler auch Infomediäre. Eine weitere strategische Aufgabe
ist die Bestimmung der Distributionslogistik. Hierbei wird definiert, ob eine Ware
gelagert wird oder in \emph{just-in-time} produziert wird, auf welchem Transportnetz und
mit welchem Service die Verteilung stattfindet.

\begin{itemize}
    \item\textbf{online-Distribution} \\
    Bei einer reinen online-Distribution findet kein materieller Güterfluss statt.
    Musikvermarkter sind ein gutes Beispiel für diese Kategorie. Bei der
    online-Distribution können aber genau wie beim materiellen Güterfluss
    direkte und indirekte Absatzkanäle gewählt werden. Auch bei open-source
    Sotwareprodukten werden häufig indirekte Kanäle in Form von Spiegelservern
    verwendet, die von anderen Organisationen als dem Hersteller betrieben werden.
    \item\textbf{offline-Distribution} \\
    Für diesen Begriff gibt es mehrere Definitionen. So wird der Begriff
    offline-Distribution verwendet, wenn in Abgrenzung zur online-Distribution,
    ein materieller Güterfluss vorliegt. Dies kann durchaus auch bei digitalen
    Gütern sinnvoll sein, um Medien mit grosser Qualität zu übertragen oder zum
    Schutz der Urheberschaft. Ausserdem redet man von offline-Distribution, wenn
    Medien innerhalb eines Intranets zur Verfügung stehen, ohne dass ein Zugriff
    auf das Internet erfolgt. Das kann bei grossen Datenmengen wie Filmen oder
    Datenbanken sinnvoll sein.
    \item\textbf{hybride Distribution} \\
    Bei der hybriden Distribution werden on- und offline-Distribution kombiniert.
    Wenn die Produkte aus materiellen Gütern bestehen, können Zusätze wie
    Handbücher, Software, Firmware und so weiter separat elektronisch verteilt
    werden. Der Materiallieferung müssen dann keine CDs, DVDs oder sonstigen
    Datenträger mehr beiliegen. Damit ist auch eine höhere Aktualität bei Updates
    gewährleistet.

    Ein grosses Thema bei digitalen Produkten ist der Urheberrechtsschutz. Ohne
    besondere Vorkehrungen hat der Hersteller keine Kontrolle über die weitere
    Herstellung und Verteilung von Kopien. Bei Fotos können digitale Wasserzeichen
    eingesetzt werden, um die Urheberschaft zu kennzeichnen. Bei Soft4 wareprodukten
    werden oft sogenannte Dongles\footnote{Es werden auch die Begriffe Kopierschutzstecker,
    Hardlock oder Key verwendet. Die eingesetzte Software kontrolliert regelmässig
    die Existenz dieses Steckers.}
    eingesetzt, die auf klassische Weise an den
    Kunden verschickt werden. Oft kann die Software nach dem Download für eine
    begrenzte Zeit ohne Einschränkungen verwendet werden und erst nach Ablauf
    einer Frist wird die Verwendung eines Product-Keys oder Dongles überprüft.
\end{itemize}

\subsubsection*{ePayment}

Die elektronische Abwicklung von Zahlungsvorgängen lässt sich folgendermassen
klassifizieren:

\begin{itemize}
    \item\textbf{Höhe des Betrages} \\
    Abweichend vom klassischen Handel, werden im Web Geschäfte abgehandelt,
    bei denen nur sehr kleine Beträge fällig werden---zum Beispiel für Webcontent.
    Das Problem bei Kleinstbeträgen ist es, den administrativen Aufwand so gering
    wie möglich zu halten damit die Kosten für die Buchungen nicht den Wert des
    Geldflusses übersteigen. Man unterscheidet bei der Höhe der Geldbeträge
    zwischen Nano-, Pico-, Micro- und Macropayment. Die genauen Betragsgrenzen
    sind nicht fix deklariert, sondern fliessend.
    \item\textbf{Zeitpunkt der Zahlung} \\
    Die Zahlung einer Dienstleistung oder eines Produktes kann vor, exakt bei
    der Übergabe oder nach der Übergabe erfolgen. Dies entspricht sinngemäss
    den drei Begriffen \emph{pre-paid} (z. B. Geldkarten), \emph{pay now}
    (z. B. Nachnahme) und \emph{pay later} (z. B. Rechnung).
    \item\textbf{Anonymität} \\
    Genau wie im klassischen Handel, gibt es auch im elektronischen Geschäft
    die Möglichkeit anonym oder nicht anonym zu bezahlen. Entscheidend ist
    dabei, ob der Käufer aufgrund des Zahlungsvorganges identifizierbar ist.
    Die klassische Bargeldbezahlung steht dabei für die anonyme Transaktion.
    Bei Kreditkarten, Rechnungen usw. kann der Käufer zumindest über mehrere
    Organisationsstufen eindeutig identifiziert werden.
    \item\textbf{Technik} \\
    Die eingesetzten Techniken können ebenso zur Differenzierung herangezogen
    werden. Die Form der Abrechnung oder der Speicherung sind mögliche Kriterien.
    Wie wird konkret abgerechnet? Werden die Geldbeträge auf Konten gutgeschrieben
    und nach gewissen Regeln in realen Währungen (was auch immer das ist)
    ausbezahlt? Bleibt der Gewinn innerhalb des Systems und kann nur gegen
    gerechnet werden - als Parallelwelt zum realen Wirtschaftssystem?
\end{itemize}

\subsection{Usability}

Der Begriff \emph{Usability} ist umfassend und wäre auch ein Kandidat für eine
Begriffsdefinition (aus der wir uns hier ja bekanntlich raushalten wollen). Es
geht um gute Anwendbarkeit, klare Orientierung und Ausrichtung auf den Benutzer
einer Applikation. Ein spezielles Problem von Webapplikationen ist die Anonymität
des Benutzers. Zu Beginn einer Sitzung haben wir kaum Informationen über Vorwissen,
Computererfahrung, Sprache und weitere Eigenschaften des Anwenders. Lediglich ein
paar technische Informationen wie Browsertyp, Betriebssystem und IP-Adresse sind
uns bekannt. Im Web wird eine grosse Anzahl potentieller Benutzer von vornherein
durch die Sprache ausgegrenzt. Mittlerweile hat man zumindest erkannt, dass Anwender
mit Behinderungen ebenfalls das Web benutzen und dass es die Effzienz steigert,
wenn auch diese Benutzergruppen integriert werden können.

Eine gute Web-Applikation sollte die Vielfalt der Anforderungen abdecken und
ergonomisch gestaltet sein. Die folgende Aufzählung sind ein paar Kriterien
ohne Anspruch auf Vollzähligkeit und sollten auf jeden Fall beachtet werden.

\begin{description}
    \item[Performance] Die Antwortzeiten sollten---auch bei langsameren
    Internetverbindungen---angemessen sein. Vermeiden sie ``schwere'' Grafiken.
    Auch wenn im Desktop-Bereich die Bedeutung der Bandbreite nicht mehr im
    Vordergrund steht, tauchen durch die Verbreitung von Smartphones und Tablets
    erneut hohe Anforderungen an eine effziente Programmierung auf.

    \item[Verfügbarkeit] Wie für \emph{Stand-Alone}-Applikationen gilt auch auf dem
    Web: Das System muss stabil laufen und sollte sich korrekt verhalten, auch
    wenn viele Benutzer gleichzeitig zugreifen.

    \item[Plattformunabhängigkeit] Die Applikation sollte auf den gängigsten
    Browsern getestet sein und fehlerfrei dargestellt werden. Präferenzen der
    Webentwickler sind dabei unwichtig. Die Zielgruppe sind die potentiellen
    Benutzer des Webauftritts. Statistische Auswertungen zeigen an mit welchen
    Clientkonfigurationen auf die Applikationen und Seiten zugegriffen wird.
    Diese Informationen können bei Optimierungen verwendet werden. Die aktuelle
    HardwarePalette wie Desktop-PCs mit grossen Bildschirmen, Laptops,
    Smartphones und Tablets sollte betriebssystemunabhängig abgedeckt werden.

    \item[Barrierefreiheit] Für blinde oder stark sehbehinderte Menschen ist
    das Internet eine gut zugängliche Informationsquelle, die Ihnen eine
    grosse Selbstständigkeit gibt. Sie verwenden Zusatzgeräte, die Ihnen die
    Informationen zum Beispiel in Brailleschrift (Punktschrift) oder durch
    Reader akustisch darstellen. Daher ist Barrierefreiheit ein grosses Thema
    in der Webentwicklung. Grafiken, Bilder, Filme sind Medien, die natürlich
    mehr auf optische Wahrnehmung ausgerichtet sind und zumindest durch die
    Verwendung \texttt{alt}- und \texttt{title}-Tags erkennbar gemacht werden sollten.

    \item[Lesbarkeit] Die Schrift sollte gut lesbar sein: Schriftgrösse und
    Kontrast auf diversen Systemen prüfen. Hintergrundgrafiken können die
    Lesbarkeit extrem vermindern.

    \item[Seitenlänge] Auf Internetseiten sollte möglichst wenig
    \emph{gescrollt}\footnote{Rollen mittels Rollbalken.} Schreiben Sie nicht:
    \texttt{Siehe weiter unten im Text}$\ldots$ sondern verwenden Sie
    lieber ber einen sogenannten \emph{Hyperlink} auf eine neue Seite. Auf der
    Hauptseite (\emph{home}) sollte \textbf{nie} gescrollt werden müssen. Vorteile von
    kurzen Seiten: Bessere Navigierbarkeit, bessere Strukturierung möglich,
    schnellere \emph{download}-Zeiten, bessere Wartbarkeit.

    \item[Orientierung] Der Benutzer sollte jederzeit sehen, wo er sich gerade
    in der Informationshierarchie befindet. Verwenden Sie immer das \texttt{<title>}-Tag
    im Seitenkopf, damit die User die Seite in den Favoriten (bzw. \emph{Bookmarks})
    einfach wieder finden können. Auf jeder Seite eine
    Navigationshierarchie\footnote{Breadcrumb-trails:
    (\href{http://psychology.wichita.edu/surl/usabilitynews/52/breadcrumb.htm}
    {http://psychology.wichita.edu/surl/usabilitynews/52/breadcrumb.htm})}
    anzugeben ist auch sinnvoll:

    \texttt{MyShop > Zahlungsmodalitäten > Rechnung > Howto}

    \item[Konsistenz] Alle Seiten einer Web-Applikation sollten sich immer
    gleichartig verhalten. Die Grafiken und Anordnungen der Steuerelemente
    sollten einheitlich sein. Verwenden Sie Fluchtlinien, um das Auge zu
    beruhigen. Beachten Sie, dass auch allfällige Bannerwerbung zum Layout passt.

    Verwenden Sie für Abstände, Rahmen, Farben, Schrift und so weiter wenn immer
    möglich CSS\footnote{Cascading Style Sheets.}-Vorlagen. Die Applikation wirkt
    einheitlicher, ruhiger und professioneller. Zudem können Sie diese viel
    einfacher \textbf{anpassen}.

    \item[Didaktik] Das System sollte leicht erlernbar sein---wenn möglich ohne
    Einführung oder Hilfe-Seiten. Alle Informationen und \emph{Links},
    die benötigt werden, um den nächsten Schritt (Warenkorb ansehen, Artikel
    zukaufen, Zahlungsbedingung aushandeln, $\ldots$) auszuwählen, sollten
    permanent ersichtlich sein.
\end{description}

Bei \emph{B2B} (bzw. \emph{B2E} Applikationen hat die \emph{Usability} natürlich
einen anderen Schwerpunkt. Hier sollte schnell gearbeitet werden können und der
User ist in der Regel ein professioneller Anwender. Oft wird eine
Einarbeitungszeit oder gar eine Schulung in Kauf genommen, um danach eine
höhere Performance zu erreichen.

\subsection{Personalisierung}

Als Personalisierung bezeichnen wir die Zuordnung von Aktivitäten und
Zugriffen zu Personen oder Gruppen. Damit können formale und inhaltliche
Merkmale abgespeichert werden. Nach der Identifikation ist eine individuelle
Ansprache möglich und die Anzeige von persönlichen Zusatzinformationen wie
Warenkorb oder Wunschliste. Eventuell kann der Kunde Inhalt und Layout
für sich verändern. Diese Informationen werden in einem Kundenprofil
abgespeichert. Einen Teil dieser Daten kann der Kunde in seinen
Profildaten selbstständig verändern und seinen Bedürfnissen anpassen.

Ausserdem kann das vorhandene Wissen über bereits getätigte Käufe und auch
über das Surfverhalten im angemeldeten Zustand innerhalb des eigenen
Webautrittes des Anbieters für Marketingaktivitäten verwendet werden.
Der Anbieter erhält so weitere Informationen, die auch für Data-Minig
verwendet werden können.

\subsection{Bannerwerbung}

Bannerwerbungen sind für die Kunden lästig. Es ist jedoch eine einfache
Möglichkeit, Werbefläche für bares Geld zu verkaufen. Die Werbefläche kann
auch per \emph{Click} oder per \emph{Show} verkauft werden.

\subsection{Passantenfunktion}

Passantenfunktionen werden alle Funktionen genannt, die Besucher eines
Webshops ohne Anmeldung ausführen können. Aus Sicht des Anbieters ist
es ein zweischneidiges Schwert. Auf der einen Seite ist die Identifizierung
eines Shopbesuchers für das eMarketing sehr wichtig. Nur durch eine genaue
Zuordnung der online-Aktivitäten zu Benutzern können detaillierte Kundenprofile
erstellt werden. Auf der anderen Seite möchten viele Shopbesucher genau
das verhindern und wünschen keine umfangreiche Analyse ihres Verhaltens
in deren Ergebnisse sie ja noch nicht einmal Einblick erhalten. Um diese
potentiellen Käufer trotzdem nicht zu verlieren und die Eintrittsschwelle
so tief wie möglich zu halten ist es in vielen Shops möglich einen direkten
Kauf zu tätigen ohne ein Kundenkonto zu eröffnen. Für den Kunden entfallen
dann diverse Vereinfachungen wie Einmalerfassung seiner Adresse, Erstellung
von persönlichen Profilen (My-Account), Übersicht über alle Bestellungen
und so weiter. Wer aber bei einem unbekannten Lieferanten voraussichtlich
nur einmal etwas kaufen möchte, hofft so eventuell nicht in die permanente
Kundenstammdatenbank aufgenommen zu werden. Wenn das Produkt per eDistribution
bezogen wird und die Zahlung mit einer anonymen Methode erfolgt, kann er
mit dieser Annahme durchaus richtig liegen. Falls aber doch eine Identifizierung
erfolgt, ist es für den Anbieter zwar aufwendiger ein Profil zu erstellen,
aber trotzdem möglich.

Für Passantenfunktionen erfolgt also kein Login und die Daten über Zahlungsart
und Lieferadresse muss der Kunde---wenn überhaupt---erst ganz am Schluss
eingeben, wenn er wirklich etwas kaufen will.


\subsection{Preisfindung}

Durch die hohe Transparenz im Internet ist die Preispolitik extrem heikel.
Ein Webshopbetreiber, der nicht bereits aufgrund seiner Bekanntheit eine
hohe Kundenbindung hat, darf sich vom Preisniveau kaum von seinen Mitbewerbern
absetzen. Es ist wichtig, die Preise für die Kunden klar erkennbar zu halten
und eventuelle Rabatte (z. B. bei grösseren Mengen oder Einkäufen über einen
bestimmten Zeitraum) möglichst früh auszuweisen. Wenn gewisse Rabatte erst
dann gewährt werden, wenn die Ware bereits im Einkaufskorb liegt, kann das für
eine längerfristige Kundenbindung zwar positiv sein, für einen Spontankauf
oder bei einem Preisvergleich mit Mitbewerbern ist das aber nicht förderlich.

\subsection{Auftragsbestätigung}

Bevor eine Bestellung wirklich ausgeführt wird, hat der Kunde die Möglichkeit,
alle Artikel, Preise, Rechnungs- und Lieferanschriften, Rabatte und so weiter
anzuschauen. Der Kunde sieht alle Produkte, die Lieferadresse und die Zahlungsart
noch einmal, bevor er zuallerletzt auf den Schalter `Auftrag versenden' klickt.

\subsection{Cross- und Up-Selling}

Mit Cross-Selling---auch Querverkauf genannt---werden die Marketingaktivitäten
bezeichnet, die ergänzende Produkte zu einem bereits ausgewählten Produkt
anbieten. Dabei kann es sich im engeren Sinne um Dienstleistungen oder
Produkte handeln, die direkt mit der Auswahl zusammenhängen, also Zubehör
oder Erweiterungen. Erweitert werden jedoch auch Produkte angeboten, die
keinen direkten Zusammenhang mit dem bisherigen Kauf haben. Zum Teil wird
dies offen kommuniziert: ``Kunden, die $x$ gekauft haben, haben auch $y$
gekauft''---ob das dann auch tatsächlich so ist, ist eine andere Sache.
In dem Zusammenhang spricht man auch von \emph{recommender}-Systemen.

Mit Up-Selling werden die Bemühungen bezeichnet dem Kunden statt dem
ausgewählten Produkt ein höherwertigeres oder zumindest teureres Produkt
zu verkaufen. Wer bei der SBB eine Bahnfahrkarte zweiter Klasse löst,
bekommt meist auch Angebote für die 1. Klasse angezeigt. Oder es werden
Mengenrabatte beim Bezug grösserer Mengen als der bisher gewünschten angezeigt.

\subsection{Data-Mining}

In grossen Warenhäusern werden Statistiken erzeugt, die Zusammenhänge im
Kaufverhalten aufdecken, um besseres Cross- bzw. Upselling zu betreiben.
Ebenso kann damit personalisiert und Aktionen können sinnvoll geplant werden.

\subsection{Warenkorb}

Der Warenkorb ist eine zentrale Funktion des Webshops. Wichtig ist zu wissen,
wie die Inhalte im Warenkorb gespeichert sind. Die Variante, die Inhalte
clientseitig zu speichern, birgt Risiken. Auf der Serverseite gibt es zwei
Varianten: a) persistent: Der Inhalt des Warenkorbes wird in einer Datenbank
gespeichert und b) transient: Der Warenkorb lebt in der \emph{Session} als
temporäre Variable. Hier muss man sich überlegen, ob die Waren auch noch
nach längerer Zeit im Korb liegen sollten. Oder macht es Sinn, die
Session-Variablen nach einer halben Stunde---mit allen Waren im
Korb---zu löschen. Die Anwenderin muss sich dann wieder neu anmelden.

\subsection{Produkte Auswahl}

Um Produkte eines Web-Shops zu finden, gibt es diverse Strategien. Wichtig
ist, dass der Kunde rasch auf das gesuchte Produkt stösst. Das kann bei
kleinen Anbietern eine einfache Tabelle mit allen im Lager befindlichen
Artikeln sein. Eine Suche nach Artikeln kann diverse Stichworte
berücksichtigen oder aber nach allen im Text vorkommenden Wörtern
suchen (Index). Häufig werden auch Produktehierarchien angeboten. So
kann sich ein Käufer wie im Supermarkt von Stockwerk zu Stockwerk
und anschliessend von Regal zu Regal bewegen, bis er beim gewünschten
Produkt ankommt.

\subsection{Glaubwürdigkeit des eMarketing}

Die Authentizität von Kundenfeedbacks, Foreneinträgen und die Überlegungen,
die hinter Cross-Selling-Aktionen stehen, sollten von kritischen Verbrauchern
immer hinterfragt werden. Zwar sind in der Vergangenheit gelegentlich
Politiker aufgeflogen, weil sie ihre Darstellungen im Internet unter einem
Decknamen, aber blöderweise von der IP-Adresse ihres Büros geschönt haben.
Man kann aber sicher sein, dass es viel professionellere Agenturen gibt,
die positive Rückmeldungen und Produktebeschreibungen lancieren, um den
Absatz zu erhöhen und eine Ware oder Dienstleistung in einem guten Licht
erscheinen zu lassen.

% ===========================================================================
\section{Informationssicherheit/Datensicherheit}
\label{sec:security}

\subsection{Ziele der Informationssicherheit}

Für die Sicherheit in der Informationstechnik müssen drei Eigenschaften erfüllt sein:

\begin{description}
    \item[Vertraulichkeit] Unberechtigte können die Informationen nicht einsehen.
    \item[Integrität] Die Inhalte sind unverfälscht.
    \item[Authentizität] Der Absender ist eindeutig erkennbar.
\end{description}

Je nach Einsatz und Anwendungsgebiet der IT-Infrastruktur ist noch eine
weitere Eigenschaft gefordert:

\begin{description}
    \item[Nichtwiderlegbarkeit/Nonrepudiation] Der Absender einer Nachricht
    kann vom Empfänger gegenüber Dritten unabstreitbar identifiziert werden.
\end{description}

\subsection{Erwartungen an die Informationssicherheit}

Eine 100\%-ige Sicherheit existiert ebensowenig wie es fehlerfreie komplexe
Systeme gibt. Der Aufwand der betrieben wird, um eine IT-Umgebung
abzusichern und um die oben genannten Ziele der Informationssicherheit
zu erfüllen hängt vom Betreiber ab. Das Bewusstsein, dass immer ein
Restrisiko vorhanden ist, ist vielleicht beunruhigend aber sicher besser
als ein arroganter Sicherheitsglaube.

\subsection{Angriffe}

Die Ziele eines Angriffes sind nur selten reine Zerstörung von Daten. Viel
häufiger ist die Kontrolle über Infrastruktur und Daten das Ziel eines
Angriffes. Bot-Netze bedienen sich einer ganzen Anzahl von unabhängigen
Rechnern, um damit wiederum Angriffe wie DoS-Attacken (siehe Abschnitt~\ref{subsub:dos},
Seite~\pageref{subsub:dos}) zu starten. Ebenso
wie bei der Spionage geht es also nicht um die Zerstörung, sondern um
einen Informations- und Kontrollgewinn. Die Urheber sind absolut nicht
an einer Entdeckung ihrer Tat interessiert und versuchen alle Spuren und
Verdachtsmomente zu vermeiden. In der Sicherheitstechnologie werden die
verschiedenen Angreifer je nach Technik und Ziel kategorisiert.

\subsubsection*{Spionage}

Um an vertrauliche oder wertvolle Firmendaten zu gelangen gibt es eine Menge
Tricks. Eine Variante ist das (illegale) herunterladen ganzer Datenbestände
von Web-Applikationen. Wie können wir uns dagegen wehren?

\subsubsection*{Passwort-Cracker/Passwort-Guesser}
\label{subsubsec:password-cracker}

Wer an einem System genügend Logins durchführen darf, kann mit einem einfachen,
jedoch zeitaufwändigen Verfahren Passwörter herausfinden. Sogenannte
\emph{Cracker}-Angriffe probieren \emph{brute-force}\footnote{``Mit aller Kraft'',
stur, durch simples Probieren aller Varianten}
alle Möglichkeiten durch. Password-\emph{Guesser} hingegen versuchen aufgrund
von Daten des Benutzers (Geburtsdatum, Namen von Verwandten, Beruf, $\ldots$)
an mögliche und sinnvolle Passwörter heranzukommen.

Meistens geschehen Passwort-Angriffe aber nicht von extern (also von ausserhalb
der Firma). Wer im Besitz einer Passwortliste ist, kann autorisierten Zugriff
auf verschiedene Konten erhalten\footnote{Das funktioniert oft auch für
verschlüsselte Passwortlisten, da diese mit sogenannten öffentlich bekannten
\emph{Hash}-Funktionen arbeiten.}.

\paragraph*{Unsichere SQL-Anfragen}
Eine spannende Art, sich Zugang zu einem Webserver ohne Rechte zu verschaffen,
funktioniert mittels unsicheren SQL-Abfragen. Häufig werden Anfragen an die
Datenbank wie folgt gestellt (hier ein PHP Beispiel):

\inputminted[firstline=1,lastline=4]{php}{query.php}

Ein schlauer Benutzer könnte nun bei der Anfrage nach seinem Benutzernamen
ins Feld einfach folgenden Eintrag tätigen.

Benutzer: \mintinline{text}{blah'; UPDATE user SET password='simple}

Falls dieser Benutzername eins-zu-eins in die Variable \$uname eingesetzt
wird, so lautet die SQL-Anfrage nun wie folgt:

\inputminted[firstline=6,lastline=10]{php}{query.php}

Der Benutzer ist zwar damit noch nicht eingeloggt, aber das Passwort wurde
bei allen Benutzern nach 'simple' abgeändert. Somit kann ein späteres
Einloggen nicht wirklich schwierig sein.

\paragraph*{Abhilfe} Abhilfe verschafft man sich, indem vor alle Apostrophe,
die vom Benutzer eingegeben wurden, ein Back-Slash \textbackslash~vorangestellt
wird\footnote{PHP kennt hier die Methode \texttt{addslashes()}}.
Somit kann ein SQL-Statement nicht mehr mutwillig beendet werden.

\subsubsection*{Horcher und \emph{The-Man-in-the-Middle}}

Ein Abhorcher (Horcher) schaut sich den ganzen Datentransfer zwischen zwei
Sockets an und versucht so, Informationen über die Schwachstellen des
Systems zu erhalten. Später kann er mit diesem Wissen das System direkt angreifen.

Ein \emph{Man-in-the-Middle} dagegen fängt den gesamten Verkehr zwischen zwei
Systemen ab, modifiziert den Inhalt und sendet die Änderungen ans Gegenüber.
Somit ist es einem \emph{Man-in-the-Middle} z.B.~möglich, einem System vorzugaukeln,
er sei ein legaler Kunde. Ein solches Einschleusen funktioniert nur bei `langsamen'
Transaktionen (z.B.~E-Mail), wo die Endpunkte nichts von der Verzögerung
(die durch die Veränderung am Inhalt entsteht) mitbekommen.

\subsubsection*{E-Shop Lifting}

Falls es einem Angreifer möglich ist, die Preise auf einer Webseite zu verändern
und so Angebote billiger zu erschleichen, so sprechen wir von E-Shop Lifting oder
auch von ``virtuellem Ladendiebstahl''\cite{ct:26:2002}. Das funktioniert z.B.~dann,
wenn der `Shop' die Preise in \emph{Hidden-Fields} auf dem Client ablegt.

\subsubsection*{Session Hijacking}

Ein Angreifer übernimmt eine bestehende Sitzung. Dieses Vorgehen wurde bei
TCP\footnote{Transfer Control Protocol.}
eingehend untersucht. Natürlich ist dies bei einfachen SessionIDs keine Hexerei.

\subsubsection*{Viren, Würmer und anderes Getier}

Viren, Würmer und Enten (Hoax) gefährden in erster Linie die Endanwender und
nicht die Web-Applikation. Es gibt jedoch immer wieder Fälle, wo auch die
Webserver mehr oder weniger gezielt attackiert werden.

\subsubsection*{DoS Attacken, Trojaner und Hintertüren}
\label{subsub:dos}

Webserver werden eher Ziel einer \textbf{Denial of Service}-Attacke (DoS) als der
allgemeine Heimanwender. Dabei machen mehrere PCs gleichzeitig simple Anfragen
an einen Webserver. Dieser Server wird dann durch die Fülle von Anfragen lahmgelegt.

Um DoS-Attacken vorzubereiten, werden häufig sogenannte
\textbf{trojanische Pferde}\footnote{Griechische Mythologie (Ilias): Die Griechen
eroberten die Stadt Troja mit Hilfe des hölzernen Trojanischen Pferdes,
in dessen hohlem Bauch sich die tapfersten Helden verbargen und so von den
ahnungslosen Trojanern in die Stadt geführt wurden.}
eingesetzt. Diese setzen, ohne das Wissen des PC-Besitzers, zu einem bestimmten
Zeitpunkt Anfragen auf das Opfer der DoS-Attacke ab.

Trojanische Pferde können aber auch eingesetzt werden, um \textbf{Hintertüren}
(sog. Backdoors ) zu öffnen. Mit offenen Hintertüren ist es einem entfernten
Angreifer möglich, alle Information über das System zu erhalten und dieses
auch nach seinen Wünschen zu modifizieren.

\subsection{Autorisierung/Authentifizierung}

Um auf einem entfernten System arbeiten zu können, braucht es Zugriff
(Authentifizierung) und Berechtigungen (Autorisierung).

Die Authentifizierung geschieht im Normalfall mit Passwörtern. Es wird
unterschieden zwischen schwacher (allein mittels Passwörtern) und
starker Authentifizierung. Letztere benötigt \emph{something to know}
und \emph{something to have} (Passwort und Streichliste). Die Autorisierung
(Bevollmächtigung mit Privilegien) geschieht nach der Authentifizierung.

\paragraph*{Definition 1 (Authentifizierung)}
Authentifizieren heisst: ``Die Echtheit von etwas bezeigen, beglaubigen``\cite{wahrig}.
In der Informatik wird ein Benutzer oder ein System (Software, Client,$\ldots$)
authentifiziert. Der Server will wissen, den Dienst in Anspruch nimmt. Dieses
Wissen über das Gegenüber erlaubt z.B.~eine Autorisierung oder eine finanzielle
Abrechnung. Zur starken Authentifizierung kann mittels Streichlisten oder
Secure-IDs vorgegangen werden.

\paragraph*{Definition 2 (Autorisierung)}
Das WAHRIG Fremdwörterlexikon umschreibt Autorisierung mit `Bevollmächtigung'.
In verteilten Systemen ist es wichtig, dass nur ermächtigte Personen Privilegien
auf bestimmten Daten erhalten: hinzufügen, suchen, ansehen, löschen, verändern,
vergeben weiterer Rechte,$\ldots$
Hier geht es darum, \textbf{was} eine Person oder ein System tun darf.

% ===========================================================================
\section{Kryptographie}

\subsection{Überblick}

Kryptographie ist die Verschlüsselung von Daten, um diese vor fremder Einsicht
zu schützen. Die Ziele der Kryptographie entsprechen denen aus
Kapitel~\ref{sec:security} auf Seite~\pageref{sec:security}.
Gemäss dem Sprichwort ``viele Wege führen nach Rom'' gibt es auch viele Verfahren,
um Informationen sicher zu übertragen. Einige dieser Methoden wollen wir in
diesem Kapitel betrachten.

Wir beginnen mit einem kurzen Einstieg zur Geschichte der Kryptographie.
Anschliessend erhalten Sie einige technische Informationen zum \textbf{XOR}-Verfahren
(siehe auch Kapitel~\ref{subsec:xor} auf Seite~\pageref{subsec:xor}) und einige
mathematische Grundlagen. Diese werden im Verfahren von Diffie/Hellmann und
RSA (siehe auch Kapitel~\ref{subsec:rsa} auf Seite~\pageref{subsec:rsa}) angewendet.
Im letzten Abschnitt (siehe auch Kapitel~\ref{subsec:java-programme} auf
Seite~\pageref{subsec:java-programme}) wird noch erklärt, wie die mathematischen
Hilfsprogramme zu bedienen sind.

Mathematische Verschlüsselungsverfahren, die auf grossen Primzahlen basieren,
haben die Stärke, dass genau berechnet werden kann, wie gross der durchschnittliche
(zeitliche, rechnerische) Aufwand sein wird, um eine Botschaft unrechtmässig
zu `entschlüsseln'.

\subsection{Geschichte}
\subsubsection*{Skytale}

Bereits vor mehr als 2500 Jahren (also ca. 5. Jh. v. Chr.) verwendeten die
Spartaner zur Übermittlung zumeist militärischer Botschaften ein aus heutiger
Sicht sehr einfaches Verschlüsselungsverfahren: die Skytale.

\begin{wrapfigure}{r}{0.3\textwidth}
    \begin{center}
        \includegraphics[width=0.25\textwidth]{images/Skytale.png}
    \end{center}
    \caption{Skytale}
\end{wrapfigure}

Benötigt wird ein runder Stab und ein Papierstreifen oder Lederstreifen.
Der Schlüssel ist der Durchmesser des Stabes. Wer einen Stab der selben
Dicke besitzt, kann die verschlüsselte Botschaft entziffern. Dieses
Verfahren gehört zu den Transkriptionsverfahren. Dabei bleiben die
Zeichen des Klartextes erhalten, werden aber in ihrer Position verändert.
Im Gegensatz zur mono- oder polyalphabetischen Substitution
(siehe auch Kapitel~\ref{subsubsec:monoalphabetische-substitution} auf
Seite~\pageref{subsubsec:monoalphabetische-substitution}), bei der die
Position bestehen bleibt und die Zeichen ersetzt werden.

\subsubsection*{Cäsar-Verfahren}

Der Algorithmus von Cäsar ist nicht viel sicherer. Das Verfahren verschiebt
die Buchstaben im Alphabet um eine vorgegebene Anzahl Buchstaben. Es kommen
26 Buchstaben im Alphabet vor, somit gibt es lediglich 26 mögliche Schlüssel.
Bedenken Sie aber, dass um 50 v. Chr. noch fast niemand lesen oder schreiben konnte.
Somit war das Verfahren für die damalige Zeit sicher genug.

\image{Caesar}{Cäsar Verschlüsselung}

\begin{beispiel}[Cäsar]
    \label{ex:caesar}
    Der Schlüssel sei 7. Das heisst, jeder
    Buchstabe wird im Alphabet um 7 Zeichen nach hinten verschoben.
\end{beispiel}

``Hallo'' $\Rightarrow$ ``Ohssv''.

\subsubsection*{Monoalphabetische Substitutionen}
\label{subsubsec:monoalphabetische-substitution}

\begin{tabular}{|c|c|c|c|c|c|c|c}
    \hline
    A & B & C & D & E & F & G &  $\ldots$ \\ \hline
    $\chi$ & $\pi$ & $\beta$ & $\mu$ & $\sigma$ & $\Phi$ & $\Omega$ &  $\ldots$ \\ \hline
\end{tabular}

Häufig wird anstelle einer Translation des Alphabetes eine beliebige
Permutation (Vertauschung) verwendet. Ob man wieder Buchstaben oder
irgendwelche kryptisch anmutenden Zeichen verwendet, spielt hier keine Rolle.

Das Verfahren kann jedoch auch sehr rasch geknackt werden. Wie wohl?

Bemerkung: Das Verfahren von Cäsar ist ein Spezialfall einer
monoalphabetischen Substitution.


\subsection{Verschlüsselungsverfahren}

Den oben genannten Verfahren ist gemeinsam, dass die Entschlüsselung einer
Botschaft mit dem gleichen Schlüssel erfolgt, mit dem auch die Verschlüsselung
durchgeführt wurde. Der Schwachpunkt dieser symmetrischen Verfahren ist die
Übertragung des Schlüssels auch wenn die eigentliche Verschlüsselung---wie
zum Beispiel beim XOR-Verfahren---sehr sicher ist. Die Lösung des Problems
ist die asymetrische Verschlüsselung, bei der ein mathematisch
zusammenhängendes Schlüsselpaar kreiert wird. Der public key muss dann
öffentlich zur Verfügung gestellt werden und dient der Verschlüsselung
durch den Absender. Der private key ist nur dem Empfänger bekannt und die
mit dem public key verschüsselte Nachricht kann nur mit Hilfe des private
keys wieder entschlüsselt werden.

Da die asymmetrische Entschlüsselung sehr rechenintensiv ist, werden die
beiden Verfahren in der Praxis gemischt eingesetzt (hybrid). Eine Botschaft
wird mit dem XOR-Verfahren verschlüsselt, der Schlüssel selber wird dann
mit dem public key des Empfängers verschlüsselt. Diese beiden verschlüsselten
Teile können dem Empfänger geschickt werden. Ohne private key kann niemand
den Schlüssel entschlüsseln und ohne Schlüssel nicht die Botschaft. Der
Empfänger macht aber genau das in der entsprechenden Reihenfolge.
Wobei dem Anwender diese Aufteilung kaum bewusst wird, denn dies
sollte von entsprechender Software automatisch erledigt werden.

\subsubsection*{Public Key}

James H. Ellis hat 1970 ein Verfahren entwickelt, bei dem Schlüssel, oder
zumindest Teile davon, öffentlich übermittelt werden können. Auch wer
diese Schlüsselteile kennt, kann Botschaften noch nicht genügend rasch
knacken. Die Idee ist brillant, doch---werden Sie sich fragen---geht
das überhaupt?

In der Regel wird bei einem solchen Public-Key\footnote{Public-Key = öffentlicher Schlüssel}
Verfahren eine Rechenoperation eingesetzt, die nicht einfach umzukehren ist.
Denken Sie z.B.~an das Wurzelziehen aus der Grundschule. Das Multiplizieren
zweier Zahlen geht rasch und einfach. Jedoch die Quadratwurzel einer Zahl
zu bestimmen, braucht schon wesentlich grösseren Aufwand. Auf einer
ähnlichen Grundidee basieren moderne Krypto-Verfahren.

Erst mittels solcher `Einwegfunktionen'\footnote{Im Gegensatz zu
Hash-Funktionen (siehe auch Kapitel~\ref{subsubsec:einfach} auf
Seite~\pageref{subsubsec:einfach}) sind diese Funktionen umkehrbar.
Jedoch ist der Aufwand, die Funktion umzukehren enorm viel höher,
als die Funktion zu berechnen.} wird es möglich, dass zwei Parteien,
die vorher noch nie miteinander in Kontakt getreten sind, geheime
Botschaften austauschen! Bisher mussten (wie beim XOR-Verfahren)
die Parteien vorher einen Schlüssel über einen geheimen Kanal verschicken!

\subsection{Das XOR-Verfahren}
\label{subsec:xor}

Die Bezeichnung XOR steht für exklusive or-Verknüpfung. Der Unterschied
zur einfachen or-Verknüpfung liegt darin, dass beim xor nur die
Verknüpfungen true ergeben, bei denen nur ein Zustand auf true
gesetzt ist. Beispiel: true or true ergibt false. XOR ist also
nichts weiter als eine boolsche Verknüpfung. Es ist keine
Verschlüsselung, obwohl häuffg von XOR-Verschlüsselung gesprochen
wird. Gemeint ist dann die Verschlüsselung unter Verwendung des
XOR-Verfahrens.

Wie funktioniert das XOR-Verfahren? Wir gehen davon aus, dass eine
Datei verschlüsselt werden soll. Diese Datei können wir einfach
in eine Byte- bzw. eine Bitfolge verwandeln. Wenn nun der Sender
und Empfänger eine zufällige Bitfolge als Schlüssel austauschen,
können der Sender Botschaft und Schlüssel bitweise mit XOR verknüpfen.
Der Empfänger kann den erhaltenen Hypher-Code mit dem Schlüssel
eindeutig wieder in die Originalbotschaft zurückwandeln.

Das XOR-Verfahren ist absolut sicher, wenn wir davon ausgehen,
dass der Schlüssel genug streut (die Null- und Einsbits sind rein
zufällig gewählt). Jetzt kann das Verfahren nicht mehr geknackt
werden. Um an die Information zu kommen, muss der Schlüssel
`geraubt' werden\footnote{Ein analoges Verfahren zum XOR-Verfahren
ist der \emph{One Time Pad} von AT\&T (1917).}.

Wichtig ist auch zu wissen, dass ein XOR-Schlüssel nur einmal
eingesetzt werden sollte.

Das XOR-Verfahren kann z.B.~auch mit einem Pseudozufallszahlen-Algorithmus
gestartet werden. Hierbei ist der Schlüssel eine sogenannte
\emph{Random-Seed}-Zahl. Wenn zwei gleich gebaute Zufallszahlengeneratoren
mit demselben Startwert beginnen, so liefern sie auch dieselbe
Zahlenfolge. Das hat den Vorteil, dass nur eine kleine Information
ausgetauscht werden muss. Das Verfahren verliert dabei aber an Sicherheit!

\subsection{Digitale Signatur}
\label{subsec:digitale-signatur}

Die digitale Signatur entspricht einer Unterschrift oder einem Siegel.
Nur wer den Siegelring (hier den Private-Key) besitzt, kann die Signatur
anfertigen.

Im Gegensatz zur Geheimhaltung bleibt beim digitalen signieren die
Botschaft unverschlüsselt. Es wird lediglich ein Hash-Code
(siehe auch Kapitel~\ref{subsec:hash} auf Seite~\pageref{subsec:hash})
der Botschaft \textbf{verschlüsselt}; und zwar diesmal mit dem
\textbf{Private-Key}.

Alle sollen die Herkunft der Botschaft überprüfen können. Hierzu wird
mit dem Public-Key der verschlüsselte Hash-Code dechiffriert und mit
dem Hash-Code der unverschlüsselten Botschaft verglichen. Da der
Public-Key öffentlich zugänglich sein soll, ist es für jede Person
möglich, die Unterschrift auf Echtheit zu prüfen; vorausgesetzt
natürlich, dass bereits dem Public-Key vertraut werden kann

\subsection{Mathematische Grundlagen zu Krypto-Verfahren}

Dieses Kapitel beleuchtet die mathematischen Hintergründe, die für
die Anwendung der Verfahren RSA, ElGamal und Diffe/Hellmann notwendig sind.

Diese Einführung erhebt keinen Anspruch auf Vollständigkeit.
Insbesondere werden wichtige Beweise weggelassen.

Es geht in den nachfolgenden Kapiteln lediglich darum, dass die
beiden Verfahren RSA und Diffie/Hellmann in groben Zügen verstanden
und angewendet werden können.

Trotz meinem pragmatischen Ansatz werden einige Grundlagen der
Zahlentheorie eingeführt:

\subsubsection*{ggT}

Der `ggT' von natürlichen Zahlen, ist der \textbf{grösste gemeinsame
Teiler} (engl. GCD = greatest common divisor).

Definition: Der grösste gemeinsame Teiler von zwei Zahlen ist die
grösste ganze Zahl, die beide Zahlen ohne Rest teilt.

\begin{beispiel}[ggT]
    \label{ex:ggt}
    Grösster gemeinsamer Teiler:
\end{beispiel}

\begin{minted}{text}
    ggT (48, 32) = 16
    ggT (10, 11) = 1
    ggT (50, 70) = 10
    ggT (35, 63) = 7
\end{minted}

\begin{bemerkung}
    Zwei Zahlen $a$ und $b$ sind genau dann teilerfremd, wenn ggT$(a, b) = 1$ ist.
\end{bemerkung}

Um den ggT von zwei grossen Zahlen zu berechnen, verwenden Sie das Programm GCD:

\begin{minted}{text}
    # java GDC
    Enter first number:735
    Enter second number:249
    GCD of given numbers is: 3
\end{minted}

\subsubsection*{Primzahlen}

Eine Primzahl ist eine Zahl, die neben sich selbst nur die 1 als
Teiler hat. Mit anderen Worten: Eine Primzahl hat genau zwei
Teiler. Die kleinste Primzahl ist 2.

\begin{beispiel}[Primzahlen]
    \label{ex:primzahlen}
    Hier die ersten Primzahlen: 2, 3, 5, 7, 11, 13, 17, 19,$\ldots$
\end{beispiel}

\begin{bemerkung}
    Ist $n$ eine beliebige positive, ganze Zahl, und $p$ eine Primzahl, so gilt:
\end{bemerkung}

\rowcolors{1}{white}{white}
\begin{tabular}{lcl}
    ggT$(n,p)$ & = & $p$, wenn $p$ Teiler von $n$ ist. \\
    & & $1$, wenn $p$ kein Teiler von $n$ ist.
\end{tabular}

Für unsere Aufgaben müssen wir Primzahlen finden---je grösser umso besser.
Beispiele von Primzahlen finden wir z.B.~auf dem Internet unter:

\href{http://www.geocities.ws/primes_r_us/small/index.html}
{http://www.geocities.ws/primes\_r\_us/small/index.html}

Natürlich werden für sehr sichere Verschlüsselungen weitaus grössere
Primzahlen (ab 300 Stellen) verwendet. Abgesehen davon, dass die
Primzahlverfahren sehr sicher sind, weisen sie noch eine zusätzliche
Stärke auf: Es ist berechenbar, mit welchem durchschnittlichen
Zeit- bzw. Rechenaufwand die Verfahren geknackt werden können.

\subsubsection*{Modulo p}

Mit Modulo (mod) bezeichnen wir das Berechnen von Divisionsresten.

$\qquad m \mod p$ := Rest, der entsteht, wenn wir $m$ durch $p$ teilen.

Beispiele:
\begin{enumerate}
    \item $17 \mod 7 = 3$ (denn $17 = 2\times 7 + 3$)
    \item $37 \mod 5 = 2$ (denn $37 = 6\times 5 + 2$)
\end{enumerate}

\begin{bemerkung}[Distributivität von Modulo-Berechnungen]
    Ist $p$ eine Primzahl, gelten folgende Regeln:
\end{bemerkung}


\begin{align}
    \label{eq:mod}
    (a + b) \mod p &= [(a \mod p) + (b \mod p)] \mod p             \\
    (a \times b) \mod p &= [(a \mod p) \times (b \mod p)] \mod p   \\
    (a)^n \mod p &= [(a \mod p)^n] \mod p
\end{align}

\begin{beispiel}[Modulo Beispiele]
    \label{ex:modulo}
\end{beispiel}


\begin{align*}
    (5 + 2)\mod 3 &= ((5\mod 3) + 2) \mod 3 = (2 + 2) \mod 3 = 4 \mod 3 = 1 \\
    (8 \times 15) \mod 7 &= (1 \times 1) \mod 7 = 1 \mod 7 = 1              \\
    (14 \times 35) \mod 3 &= (2 \times 2) \mod 3 = 4 \mod 3 = 1             \\
    46^{10} \mod 11 &= 2^{10} \mod 11 = 4^5 \mod 11 = (4\times 4^4) \mod 11 = (4\times 16^2) \mod 11 \\
    &= (4\times 5^2) \mod 11 = (4\time 25) \mod 11 = (4\times 3) \mod 11 = 1
\end{align*}

\begin{bemerkung}[RSA]
    Für das RSA-Verfahren (siehe auch Kapitel~\ref{subsec:rsa} auf Seite~\pageref{subsec:rsa})
    und das Verfahren von Diffie/Hellmann (siehe auch Kapitel~\ref{subsec:diffie-hellmann}
    auf Seite~\pageref{subsec:diffie-hellmann}) brauchen wir $a^b \mod c$ zu berechnen.
\end{bemerkung}

\begin{Exercise}[%
title={Modulo Berechnung},
label={exercise:modulo}]
\end{Exercise}
Berechnen Sie zum Beispiel $23^{17} \mod 7$. Erstellen Sie hierzu ein kleines
Hilfsprogramm in der Programmiersprache Ihrer Wahl (z.B.~Java). Der Aufruf dieses
Hilfprogrammes könnte wie folgt aussehen:
\begin{minted}[autogobble]{text}
    >java AhBmC 23 17 7
\end{minted}


\subsubsection*{Inverses Modulo $p$}

Für das RSA-Verfahren benötigen wir noch die folgende Rechenoperation.
Wenn Sie sich nur für Difie-Hellman oder das Verfahren von ElGamal
interessieren, so können Sie dieses Kapitel überblättern.

Wenn wir Modulo $p$ rechnen und $p$ eine Primzahl ist, so gibt es für
jede Zahl $m$ ein sogenanntes multiplikatives inverses Modulo $p$.
Das heisst: für jedes $m$ gibt es ein $n$, so dass $(m\times n) \mod p = 1$ ist.

\begin{align*}
    p = 7, m = 3 &\rightarrow n = 5~ \text{(denn~} 3\times 5\mod 7 = 1 \text{)} \\
    p = 7, m = 6 &\rightarrow n = 6~ \text{(denn~} 6\times 6\mod 7 = 1 \text{)} \\
    p = 11, m = 5 &\rightarrow n = x
\end{align*}

Hier ein simples Vorgehen, um das Inverse ($\mod p$) zu finden:

Nehmen wir z.B. $p = 13$ und $m = 5$

\begin{align*}
    5 \times 2 \mod 13 &= 10 \mod 13 = 10 \\
    5 \times 3 \mod 13 &= 15 \mod 13 = 2  \\
    5 \times 4 \mod 13 &= 20 \mod 13 = 7  \\
    5 \times 5 \mod 13 &= 25 \mod 13 = 12 \\
    5 \times 6 \mod 13 &= 30 \mod 13 = 4  \\
    5 \times 7 \mod 13 &= 35 \mod 13 = 9  \\
    5 \times 8 \mod 13 &= 49 \mod 13 = 1  \\
\end{align*}

Daraus ergibt sich 8 als das Inverse ($\mod 13$) zu 5,
denn $5\times 8 \mod 13 = 1$.

\begin{Exercise}[%
title={Inverse Modulo Berechnung},
label={exercise:inverse-modulo}]

    Erstellen Sie hierzu ein kleines Hilfsprogramm in der Programmiersprache
    Ihrer Wahl (z.B.~Java). Der Aufruf dieses Hilfprogrammes könnte wie folgt aussehen:
    \begin{minted}[autogobble]{text}
        >java MInv 5 13
    \end{minted}
\end{Exercise}

\subsubsection*{Einfach \& Schwierig}
\label{subsubsec:einfach}

Es gibt nun zwei Eigenschaften, die die Primzahlverfahren sehr sicher machen:
\begin{enumerate}
    \item Die Zerlegung von groÿen Zahlen in ihre Primfaktoren ist schwierig,
    die Multiplikation dagegen ist einfach.

    Sind $p$ und $q$ zwei 300-stellige Primzahlen, so ist $p\times q$ einfach
    zu berechnen; die Primfaktorzerlegung einer 600-stelligen Zahl hingegen
    ist sehr zeitaufwändig (`pröbeln'). Genau diese Schwierigkeit nutzt
    das RSA Verfahren.

    \item Exponenten Modulo einer Primzahl zu rechnen ist einfach.
    Die Umkehrung (den sog. diskreten Logarithmus) zu finden ist schwierig.
    Diese Schwierigkeit wird vom Diffie/Hellman-Verfahren wie auch vom
    ElGamal-Algorithmus ausgenutzt:

    $a^b \mod p$ zu berechnen ist einfach (siehe Beispiel~\ref{ex:modulo})

    Aus der Gleichung $a^n \mod p = s$ das $n$ zu berechnen, ist hingegen schwierig.
\end{enumerate}

\subsection{Hashfunktionen}
\label{subsec:hash}

\subsubsection*{Standard-Hashfunktionen}

Hashfunktionen sind schnell zu berechnende Schlüsseltransformationen oder
auf deutsch auch Streuwertfunktionen genannt. Sie bilden mittels einer
Funktion Daten---wie zum Beispiel Passwörter, Schlüssel, Texte (lang
und kurz), beliebige digitale Objekte, E-Mails und so weiter---auf
einen vergleichsweise kleinen Wertebereich mit einheitlicher Grösse
ab. So wird aus einem grossen Objekt eine möglichst kleine Datenstruktur
aus nur wenigen Bytes.

Hashfunktionen gehören zu den sogenannten Einwegfunktionen. Bei
schwachen Hashfunktionen können zwar mehrere Ausgangstexte oder
Urbilder auf den gleichen Hashcode verweisen\footnote{Das nennt sich dann
Kollision} ich kann aber nicht von einem Hashwert auf den
Ursprungstext zurückrechnen. Das ist unabhängig von Rechnerausstattung
und Geschwindigkeit.

Eine Eigenheit von herkömmlichen Hashfunktionen ist es, dass der
gleiche Text bei jeder Anwendung der Funktion auf einen identischen
Hashcode kommt. Zwei gleiche Passwörter liefern mir also auch den
gleichen Hashcode. Das wird mit der Erstellung und Verwendung von
\emph{rainbow-tables}, ausgenutzt. Das ist eine Sammlung von
Hashcodes, deren Ausgangstexte bekannt sind.

Ein simples Beispiel einer Hashfunktion ist die Quersummenbildung,
auch wenn Sie viele Kriterien einer guten Hashfunktion nicht
erfüllt. Damit wird eine belebig lange Zahlenreihe auf genau
eine Ziffer reduziert. Für die meisten Anwendungen ist diese
Funktion aber deutlich zu einfach, denn gute Hasfunktionen sollten
die folgenden Eigenschaften erfüllen:

\begin{tabular}{p{0.25\textwidth}|p{0.7\textwidth}}
    \hline
    schnell berechenbar &
    Eine Hashfunktion soll sehr schnell berechnet werden können. \\ \hline

    kleiner Wertebereich &
    Objekte beliebiger Grösse werden auf wenige Bytes abgebildet. Java
    verwendet eine Hashfunktion für Strings, die jede Zeichenkette
    auf lediglich 4 Bytes abbildet. \\ \hline

    nicht umkehrbar &
    Hash Funktionen können nicht rückgängig gemacht werden. Es handelt
    sich hier um eine Art Einwegfunktion. Jedoch nicht so, dass die
    Funktion sehr schwierig umzukehren ist, wie dies bei den
    Verschlüsselungsverfahren der Fall ist, sondern, dass die
    Funktion überhaupt nicht umzukehren ist. Aus dem Hash-Wert
    können die ursprünglichen Daten nicht wieder rekonstruiert werden.

    Das ist übrigens eine einfache Konsequenz aus obiger Tatsache
    des kleinen Wertebereiches. Mehrere Objekte können denselben
    Hash-Wert erhalten. \\ \hline

    gute Streuung &
    Hashfunktionen sollen im Wertebereich stark streuen. Das heisst:
    Zwei unterschiedliche Objekte sollten mit grosser Wahrscheinlichkeit
    zwei verschiedene Hash-Werte liefern. \\ \hline
\end{tabular}

Einsatzgebiete:
\begin{itemize}
    \item Digitale Signatur (siehe auch Kapitel~\ref{subsec:digitale-signatur}
    auf Seite~\pageref{subsec:digitale-signatur})
    \item Digitaler Fingerabdruck
    \item Prüfsummenbildung
    \item Passwortlisten (siehe auch Kapitel~\ref{subsubsec:password-cracker}
    auf Seite~\pageref{subsubsec:password-cracker})
    \item Indexierung von Textattributen in Datenbanken
    \item Hash-Tabellen (nicht Inhalt dieses Kurses.)
\end{itemize}

\begin{Exercise}[%
title={Hash-Code Berechnung},
label={exercise:hash-code}]

    Erstellen Sie hierzu ein kleines Hilfsprogramm in der Programmiersprache
    Ihrer Wahl (z.B.~Java). Der Aufruf dieses Hilfprogrammes könnte wie folgt aussehen:
    \begin{minted}[autogobble]{text}
        >java Hash "Geheimnachricht" "Geheim Nachricht"

        Hashcode of 'Geheimnachricht' = '0x4A70B2C1'
        Hashcode of 'Geheim Nachricht' = '0xAB60729F'
    \end{minted}
\end{Exercise}


\subsubsection*{salted Hash}

Die einzige anerkannte Methode um vom Hashwert wieder auf sein Urbild
zu kommen sind die oben erwähnten \emph{rainbow-tables}. Sehr lange
und aus verschiedenen Zeichentypen bestehende Passwörter wird man
mit grosser Sicherheit auch so nicht finden, aber welcher Administrator
wird sich da schon auf das Sicherheitsverständnis der Benutzer
verlassen - zumal auch die Passwortwahl von Administratoren nicht
über alle Zweifel erhaben ist.

Ein Ausweg sind sogenannte Salts. Das sind zufallsgenerierte Codes,
die dem Ursprungstext hinzugefügt werden, bevor die Hashfunktion
angewendet wird. Der Salt muss natürlich ebenfalls abgespeichert
werden. Aber der Aufwand um jetzt alle Passwortkombinationen mit
dem Salt durchzutesten ist riesig.

\subsection{Das Verfahren von Diffie/Hellmann}
\label{subsec:diffie-hellmann}

\image{DHVerfahren}{Das Diffie Hellmann Verfahren}

Das Verfahren von Diffie/Hellmann (DH-Verfahren\footnote{Whitfield Diffie,
Martin Hellman und Ralph Merkle} 1976) erlaubt es, Daten verschlüsselt
zu übermitteln, ohne vorab einen geheimen Schlüssel auf einem separaten
Kanal zu transportieren\footnote{Das Verfahren wurde schon vor Diffie und
Hellman von Malcolm Williamson vorgeschlagen.}.

\subsubsection*{Ausgangslage}

\begin{itemize}
    \item Ausgangssituation: Alice (A) und Bob (B) wollen Nachrichten über
    eine öffentlich zugängliche Leitung austauschen. Diese Leitung wird
    möglicherweise abgehorcht.
    \item Alice und Bob hatten vorher noch keine Schlüssel miteinander ausgetauscht.
    \item Das Diffie/Hellman-Verfahren erlaubt es, eine Botschaft über eine
    Leitung zu übermitteln, ohne dass ein `Horcher' die Nachricht verstehen kann.
    \item Es wird davon ausgegangen, dass der `Horcher' sich nicht als Sender
    oder Empfänger ausgeben kann. (Die Transaktion sei zu schnell, als
    dass die \emph{Man-in-the-Middle}-Attacke funktionieren könnte.)
\end{itemize}

\subsubsection*{Vorgehen im Diffie/Hellman-Verfahren}

Das Diffie/Hellman-Verfahren besteht aus folgenden Schritten:
\begin{enumerate}
    \item Die Kommunikationspartner A und B (Alice und Bob) entscheiden sich
    gemeinsam für eine grosse Primzahl $p$. (Je grösser die Primzahl, umso
    sicherer das Verfahren.)

    \item A und B suchen eine Zahl $s$, die kleiner als $p$ ist
    (Bedingung: $1 < s < p$)\footnote{Genaugenommen sollte $s$ eine sogenannte
    Primitivwurzel modulo $p$ sein (Siehe~\cite{buchmann}). Das Verfahren
    funktioniert auch für andere Zahlen, kann aber unter Umständen einfach werden.}.
    Wichtig: Die beiden Zahlen $p$ und $s$ können unverschlüsselt über die
    Telefonleitung übermittelt werden; sie sind also öffentlich zugänglich.
    Cleo (C) in der Mitte hört $p$ und $s$, kann aber mit diesen Zahlen
    noch nichts anfangen.

    \item Alice sucht sich im Geheimen eine Zahl $a$; Bob sucht sich im Geheimen
    eine Zahl $b$; Die beiden Zahlen sollten kleiner als $p − 1$ sein.

    \item A berechnet $m := s \mod p$ und schickt das Resultat an B;
    B berechnet $n := s \mod p$ und schickt das Resultat an A;
    Cleo hört zwar $m$ und $n$ mit und kann damit nun theoretisch $a$ und $b$
    berechnen. Er braucht dazu aber viel zu lange.

    \item A berechnet die Geheimzahl $g = n \mod p$;
    B berechnet die Geheimzahl $g = m \mod p$.

    Bemerkung $g = n^a = m^b \text{, denn } n^a = {(s^b)}^a = s^{ba} = s^{ab} = {(s^a)}^b = m^b$

    C kann $g$ nicht in nützlicher Frist berechnen. Das einzige mathematische Verfahren für C,
    um $g$ zu `berechnen', heisst \emph{ausprobieren}.
    \footnote{Es gibt einige mathematische Tricks (das Babystep-Giantstep-Verfahren von
    Shanks, der Pollard-ρ-Algorithmus oder das Verfahren von Pohlig-Hellman), um schneller
    zum Ziel zu kommen, und es gibt Fälle von Primzahlen, bei denen das Verfahren schneller
    werden kann als stures ausprobieren.}.

    \item A und B können nun ihre Nachrichten mit $g$ verschlüsseln und entschlüsseln
    (z.B.~mit dem XOR-Verfahren (siehe auch Kapitel~\ref{subsec:xor}auf Seite~\pageref{subsec:xor})).
\end{enumerate}

%\subsubsection*{Ein Zahlenbeispiel}

\subsection{Das Verfahren von Rivest, Shamir und Adleman (RSA)}
\label{subsec:rsa}

Das RSA-Verfahren wurde von Ronald R. Rivest, Adi Shamir und Leonard M.
Adleman entwickelt.

Das RSA-Verfahren ist ein \emph{Public Key} Verfahren. Das heisst, der Algorithmus
und der Schlüssel zum Verschlüsseln (codieren, chiffrieren) von Botschaften wird
öffentlich bekannt gegeben. Nur der Schlüssel zum Entschlüsseln (decodieren,
in Klartext zurückverwandeln) der Botschaften wird geheim gehalten.

Das RSA-Verfahren basiert auf der Tatsache, dass es sehr einfach ist, zwei
Primzahlen miteinander zu multiplizieren, dass es aber äusserst aufwändig ist,
die beiden Primzahlen wieder zu finden, falls nur noch das Produkt bekannt ist.

\subsubsection*{Vorgehen im RSA-Verfahren}

Das RSA-Verfahren besteht aus vier Schritten.

\begin{enumerate}
    \item Der Empfänger generiert ein Schlüsselpaar: einen privaten (geheimen)
    und einen dazu passenden öffentlichen Schlüssel,
    \item der Empfänger publiziert den öffentlichen Teil,
    \item der Sender verschlüsselt mit dem öffentlichen Schlüssel des Empfängers
    seine Botschaft und
    \item der Empfänger entschlüsselt die Botschaft mit seinem geheimen Schlüssel.
\end{enumerate}

Der Trick dabei ist, dass nur der Empfänger die Botschaft in sinnvoller Zeit
entschlüsseln kann, denn nur er kennt die Primfaktorzerlegung des Schlüssels,
wie wir gleich sehen werden. Nun aber die 4 Schritte im Detail:

\paragraph*{Schritt 1: Schlüssel generieren}
In diesem Schritt generiert der Empfänger einen öffentlichen Schlüssel.
Alle können danach mit diesem Schlüssel Botschaften chiffrieren (verschlüsseln),
aber nur der Empfänger kann sie wieder dechiffrieren (entschlüsseln).

\renewcommand{\theenumi}{\alph{enumi}}
\begin{enumerate}[label={\alph*)}]
    \item Der Empfänger sucht zwei (möglichst grosse) Primzahlen $p$ und $q$.
    \item Der Empfänger berechnet $r = p\times q$.
    \item Der Empfänger berechnet zudem $s = (p − 1) \times (q − 1)$.
    \item Der Empfänger bestimmt ein beliebiges $c$ mit den beiden folgenden
    Eigenschaften: $c < s$ und ggT$(c, s) = 1$. Das erreicht der Empfänger
    zum Beispiel einfach, indem er eine Primzahl sucht, die kleiner als $s$
    ist. Hier kann das Programm GCD eingesetzt werden:

    \begin{center}
        \texttt{>java GCD c s}
    \end{center}

    muss 1 ergeben!
\end{enumerate}


\paragraph*{Schritt 2: Veröffentlichen des Schlüssels}
Der Empfänger gibt $r$ und $c$ als öffentlichen Schlüssel bekannt.
Zum Beispiel steht auf seiner Homepage vereinfacht: \emph{Der öffentliche
Schlüssel von daniel.senften@talent-factory.ch ist (r = 289073 und c = 353)}.


\paragraph*{Schritt 3: Verschlüsseln einer Botschaft}


\subsubsection*{Ein Zahlenbeispiel}

\subsection{Der eigene öffentliche Schlüssel}

\subsection{Java Hilfsprogramme}
\label{subsec:java-programme}

\subsubsection*{\texttt{XorKryptRandom}}
\subsubsection*{Grösster gemeinsamer Teiler: \texttt{GCD}}
\subsubsection*{Das Inverse modulo einer Primzahl: \texttt{MInv}}
\subsubsection*{Potenzieren modulo einer Primzahl: \texttt{AhBmC}}

% ===========================================================================
\section{Implementierung}

\subsection{Anpassungen - Change Management}

\subsection{Session}

% ===========================================================================
\section{Übungen und Aufgaben}

\subsection{Warenkorb}
\subsubsection*{Alternative: Neuer Shop}

\subsection{Shop-Vergleich}

\subsection{Sicherheit}
\subsubsection*{Schutz gegen Angriffe}
\subsubsection*{Angriff}

\subsection{Verschlüsselung}

\subsection{Verschlüsselung - Praxis}

\subsection{Standard Web-Shops}

% ===========================================================================
\section{GnuPG}

\subsection{Installation}

\subsection{GPG-Home Verzeichnis}

\subsection{Schlüssel generieren}

\subsection{Importieren von Schlüsseln}

\subsection{Schlüssel unterschreiben und beglaubigen}

\subsection{Verschlüsseln / Entschlüsseln einer Botschaft}

