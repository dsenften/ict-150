\mode*

\setmonofont{Verdana}

% ===========================================================================
\section{Modulidentifkation}

\definecolor{mydarkgrey}{RGB}{96,96,96}
\definecolor{mygrey}{RGB}{224,224,224}

\rowcolors{1}{mygrey}{white}

\begin{center}
    \begin{tabular}[h]{|p{4cm}|p{12.5cm}|}
        \hline
        Modulnummer & 150 \\ \hline
        Titel & E-Business-Applikationen anpassen \\ \hline
        Kompetenz & E-Business-Applikationen gemäss Vorgabe und unter Beachtung der
        Sicherheitsvorschriften und technischer Rahmenbedingungen anpassen. \\ \hline
        \multirow{1}{*}{Handlungsziele} &
        Aufbau der Applikation, Transaktionskonzept, Applikationsumgebung
        und Rahmenbedingungen (Sicherheit, Performance, Verfügbarkeit, Transaktionsvolumen, usw.) erfassen. \\
        \cellcolor{white}  & Vorgabe analysieren, clientseitigen, serverseitigen und datenbankseiti-gen
        Änderungsbedarf formulieren. \\
        \cellcolor{white}  & Auswirkungen der Änderungen auf Sicherheit und Schutzwürdigkeit der
        Informationen bei allen beteiligten Komponenten wie Client, Webserver, Applikationsserver
        und Datenbankserver überprüfen und dokumentieren. \\
        \cellcolor{white}  & Änderungen inklusive Implementierung und Test (funktional und nicht-funktional)
        gemäss einem vordefinierten Änderungsprozess planen. \\
        \cellcolor{white}  & Änderungen realisieren, testen und dokumentieren. \\ \hline
        Kompetenzfeld & Web Engineering \\ \hline
        Objekt & Online-Shop, Ticketing, Wiki oder E-Learning Lösung. \\ \hline
        Niveau & 4 \\ \hline
    \end{tabular}
\end{center}

% ===========================================================================
\section{Einleitung und Rahmenbedingungen}

\subsection{Begriffsdefnitionen - Modul 150}

Mit dem Präfix 'E' versehen, geistern in der Informatik viele Begriffe herum.
Auch in der Modulidentifikation zu diesem Modul werden die Begriffe \emph{E-Business}
und \emph{E-Commerce} parallel verwendet. Je nach verwendeter Literatur, werden
zwischen diesen beiden Begriffen aber wesentliche Unterschiede gemacht.
In diesem Skript werden wir uns aus mühsamen Begriffsdefinitionen möglichst
heraus halten und einen Überblick über Techniken und Grundlagen geben,
die bei der Anpassung und beim Entwurf von Webapplikationen eine Rolle
spielen. Wenn hier im Skript der Begriff \emph{E-Business} verwendet wird,
bezeichnet er Applikationen, die mindestens die Komponenten Produktauswahl,
Zahlungsverkehr und Warentransport (elektronisch oder physikalisch) enthalten.

Soziale Plattformen, Foren und so weiter gehören nicht zu dieser
Applikationsgruppe. Die meisten hier betrachteten Techniken werden
auch bei diesen Anwendungen eingesetzt.

Sie sollten in diesem Modul einen Überblick über Themen bekommen, die im
Bereich \emph{E-Business} wichtig sind. Die können in 40 Lektionen zwar nicht
vollständig bearbeitet werden, aber eine Sensibilisierung auf mögliche
Probleme und der Erwerb von Grundkenntnissen mit zusätzlicher selbstständiger
Vertiefung sind sicher möglich.

\subsection{Projektmanagement}

Die Abwicklung von Webprojekten und die Realisierung von E-Business Lösungen
erfordert ein professionelles Projektmanagement. Welche Methoden dabei
eingesetzt werden, ist von den beteiligten Unternehmen und von der Anwendung
abhängig und nicht Bestandteil dieses Moduls. Lediglich der Begriff \emph{Change
Management} sei hier noch erwähnt, auf den wir später kurz eingehen werden.


\subsection{Partnerrollen bei Webapplikationen}

Auch wenn die Begriffsdefinitionen hier nicht im Vordergrund stehen, müssen
wir eine Reihe von Merkmalen kennen, da sie die Applikation und auch die
eingesetzte Technik stark beeinflussen. Wir müssen uns klar machen, welche
Unterschiede in Bezug auf die Geschäftspartner bei Webapplikationen bestehen
und in welchem Gebiet unsere Applikation eingesetzt wird. Das hat Konsequenzen
auf den Entwurf, die anzuwendenden Gesetze, die Technik und die Gestaltung
der Applikation.

Anmerkung bezüglich der Partnerrolle 'C': In der Literatur
und auch im Internet finden wir die Bezeichnungen \emph{customer} und \emph{consumer}.
In Verbindung mit der Rolle 'A' für \emph{administration} bedeutet es dann
\emph{citizen} also Bürger.

\subsubsection*{B2C - Business to Customer/Consumer}

Diese Geschäftsbeziehung fällt einem vermutlich als erste mögliche Beziehungsvariante
ein. Sie entspricht dem klassischen Ladengeschäft, das seine Waren an Endkunden
verkauft. In der elektronischen Variante ist dazu kein physikalischer Laden mehr
nötig, aber viele der gesetzlichen Bestimmungen zum Thema Konsumentenschutz, sind
mittlerweile auch im Internet-Handel gültig. Etwas komplizierter wird es durch die
Globalisierung via Internet. Wenn zwischen den Ländern des Verkäufers und des Käufers
Gesetze und Vereinbarungen bestehen, ist die Unsicherheit zwar limitiert, aber Zölle,
Einfuhrbeschränkungen auf bestimmte Güter, Versandmodalitäten und eventuell auch
sprachliche Schwierigkeiten sorgen für ausreichend Möglichkeiten einen Geschäftsvorgang
kompliziert werden zu lassen. Was die juristischen Anforderungen zum Thema
Konsumentenschutz betrifft, sind hier besonders das Rückgaberecht und die
Auszeichnungspflicht (Angabe des Warenpreises inklusive Mehrwertsteuer) zu nennen.
Diese Konstellation der Geschäftspartner entspricht auch meistens den von uns
untersuchten WebshopAnwendungen.

\subsubsection*{B2B - Business to Business}

Hierbei handelt es sich um die Beziehung zweier Geschäftspartner. Beide Partner
sind sich im Normalfall bekannt. Zumindest muss eine Überprüfung der Autorisierung
erfolgen, um sicher zu stellen, dass es sich bei beiden Partnern um Firmen handelt.
Vor unberechtigten Zugriffen durch Endkunden muss die Anwendung geschützt werden.
Bei Verkaufsgeschäften bestehen keine Rücknahmepflichten und die Preisauszeichnung
erfolgt ohne Angabe der Mehrwertsteuer. Ein privater Kunde hat andere Anforderungen
und Erwartungen an eine Applikation als ein Lieferant oder Zwischenhändler.
Zumeist wird diesen mehr Können in Bezug auf \emph{Usability} und Ziele der
Applikation zugemutet als es bei anonymen Endkunden möglich ist. So kann die
Schnittstelle durchaus einmal aus einem autorisierten
FTP\footnote{File Transfer Protocol}-Download einer Datenbank bestehen.

\subsubsection*{C2C - Customer to Customer}

Das wohl erfolgreichste Beispiel für diese Kundenbeziehung ist EBay\textsuperscript{TM}.
Ursprünglich ging es hier um einen Gebrauchtwarenhandel, der zu Beginn
ausschliesslich zwischen Privatpersonen durchgeführt wurde. Aktuell tummeln
sich aber auch jede Menge Händler auf der Plattform und bieten Neuwaren an.
Allerdings habe diese die Verpflichtung sich klar als solche zu kennzeichnen
und die Bedingungen entsprechen dann einer B2C-Beziehung.

Findige (und zumeist auch windige) Juristenbüros haben diesen Umstand
ausgenutzt und untersuchen die Anbieter nach ihren Artikel- und Angebotsprofilen.
Wer eine alte CD-Sammlung auflöst, verkauft verschiedene CDs einmal und selten
vielleicht zweimal. Wer aber die gleiche CD mehrfach verkauft muss sich eine
gute Argumentation zurechtlegen, warum er sich nicht als Händler ausgibt
und muss entsprechend mit Abmahnungen rechnen (siehe auch
Kapitel~\ref{subsec:juristische-grundlagen} auf Seite~\pageref{subsec:juristische-grundlagen}).

\subsubsection*{A2C - Administration to A, B, C}

Es gibt eine ganze Reihe von eBegriffen, die im Bereich von Behördenapplikationen
verwendet werden. eGovernment, eAdministration, eJustice und eDemocracy sind
einige davon. Interessant ist die derzeitige Diskussion in den Diskussionsforen
der deutschen Wikipedia (zum Beispiel beim Begriff E-Government). Es wird
wohl noch einige Zeit verstreichen, bis sich feste Begriffe mit eindeutigen
Definitionen durchgesetzt haben. Generell geht es aber um Applikationen,
die Bürgern, Firmen oder anderen Behörden angeboten werden, um den Kontakt
auf elektronischem Weg zu ermöglichen. Das können stark sicherheitsrelevante
Anwendungen wie Steuerabrechnungen, E-Voting (wählen und abstimmen über das
Internet) und Ausschreibungen sein, aber auch einfache Applikationen, die den
Gang zur Verwaltung ersparen.

Ein Beispiel für eine \emph{Administration to Business}-Applikation finden Sie
auf der Website des
\href{https://www.blw.admin.ch/blw/de/home.html}{Bundesamts für Landwirtschaft BLW}.
Dort werden Einfuhrkontingente von landwirtschaftlichen Erzeugnissen wie Fleisch,
Wurstwaren, Zuchtrinder, Kartoffelprodukte periodisch versteigert. Allerdings
richtet sich die Web-Applikation eVersteigerung nicht an Gelegenheitsnutzer,
sondern erfordert eine vorherige Anmeldung und die Installation eines Zertifikates.
Anschliessend können Gebote für die ausgeschriebenen Importprodukte abgegeben
werden. Für die Bieter ist das nicht immer transparent und führt gelegentlich
zu Aufruhr, wie im Bericht ``Import von Filets ist eine Lotterie''
\footnote{Bericht im Tages-Anzeiger vom Montag, 08. Februar 2010} nachzulesen ist.

\subsubsection*{B2E - Business to Employee}

Anwendungen, die sich an die Angestellten einer Firma wenden, unterliegen
ebenfalls Bedingungen zum Datenschutz und auch zur Datensicherheit.
Besonders der Zugang von unautorisierten Personen ist hier ein Thema, denn
sonst wäre es ja keine reine B2E-Applikation mehr. Diese Applikationen
laufen meistens auf dem Intranet, da sie am Arbeitsplatz benötigt werden,
sie können aber teilweise auch via Internet sinnvoll sein, wie zum Beispiel
ein spezieller Webshop, der sich nur an Mitarbeiter richtet. Diese Kategorie
ist in der Übersicht nicht vertreten.

\rowcolors{2}{mygrey}{white}
\begin{table}
    \begin{center}
        \begin{tabular}[ht]{|p{3.5cm}|p{4cm}|p{4cm}|p{4cm}|}
            \hline
            \cellcolor{orange!25} &
            \cellcolor{orange!25}\textbf{Administration} &
            \cellcolor{orange!25}\textbf{Business} &
            \cellcolor{orange!25}\textbf{Consumer} \\ \hline
            \cellcolor{orange!25}\textbf{Administration} & \textbf{A2A} & \textbf{A2B} & \textbf{A2C}      \\
            \cellcolor{orange!25}                  & Applikation zwischen Behörden (z.B. Interpol)
            & Applikationen für öffentliche Ausschreibungen &      \\ \hline
            \cellcolor{orange!25}\textbf{Business} & \textbf{B2A} & \textbf{B2B} & \textbf{B2C}  \\
            \cellcolor{orange!25}                  & & & Webshops; Anbieter sind gewerbsmässige Unternehmen  \\ \hline
            \cellcolor{orange!25}\textbf{Consumer} & \textbf{C2A} & \textbf{C2B} & \textbf{C2C}  \\
            \cellcolor{orange!25}                  & & Angebotssuche potentielle Kunden schreiben ihre Projekte aus, Firmen bieten an
            & Kleinanzeigen, Flohmärkte, alle privaten Handelsmöglichkeiten \\ \hline
        \end{tabular}
    \end{center}
    \caption{Anwendungsbereiche des eBusiness}
\end{table}


\subsection{Juristische Grundlagen}
\label{subsec:juristische-grundlagen}

Aufgrund der vermeintlichen (aber definitiv nicht realen) Anonymität im
Internet und der automatischen Globalisierung von Webauftritten, wird das
Internet von vielen als rechtsfreier Raum angesehen. Das kann aber
spätestens dann zu einem bösen Erwachen führen, wenn man eine Abmahnung
oder gar eine Klage erhält. Zumindest mit den wesentlichen rechtlichen
Begriffen und Gepflogenheiten muss man sich als Webentwickler auskennen.
Dabei ist es gleichgültig, ob es sich bei den erstellten Auftritten um
professionelle oder um private Seiten handelt.

\subsubsection*{Abmahnungen}

Bei Abmahnungen handelt es sich ursprünglich um Verfahrensvereinfachungen im
Wettbewerbsrecht. Wer einen Verstoss im Wettbewerbsrecht feststellt, war damit
nicht mehr gezwungen auf dem gerichtlichen Weg vorzugehen, sondern konnte
aussergerichtlich direkt mit der anderen Partei Kontakt aufnehmen und den
Fall lösen. Erst wenn sich die Partner nicht einigen können werden die
Gerichte damit belastet. Bei offensichtlichen Verstössen und einsichtigen
Konfliktparteien, können damit nicht nur die Gerichte entlastet werden, sondern
auch die Folgekosten für die widerrechtlich handelnde Partei stark reduziert
werden.

Soweit die grundsätzlich positive Idee dieses Rechtsmittels.

Leider wird dieses Instrument auch stark missbraucht. Es hat sich als ein
einträgliches Geschäft erwiesen, im Internet nach kleinen Verstössen zu
fahnden und die Urheber einfach abzumahnen, in der Hoffnung, dass diese--bei
relativ geringen Abmahngebühren--den Weg des geringsten Widerstandes nehmen
und bezahlen. Ein grosser Teil der Abmahngründe ist dabei juristisch nicht
haltbar. Natürlich hat sich auch eine Gegenbewegung gebildet, die bei Bedarf
Hilfe anbietet. Zum Beispiel \href{http://www.abmahnwelle.de}{http://www.abmahnwelle.de}.

\subsubsection*{Impressum}

Ein Impressum ist die Angabe von verantwortlichen Personen und deren Kontaktdaten.
Den Ursprung hat das Impressum bei Druckereierzeugnissen wie zum Beispiel
Zeitungen, Flugblättern und Werbeprospekten. Dort muss nicht nur der verantwortliche
Redakteur genannt werden, sondern auch die Adresse des Medienunternehmens und
eventuelle Beteiligungen, um mehr Transparenz in Bezug auf wirtschaftliche
Verflechtungen zu erhalten. In der Schweiz gilt die Impressumspflicht generell
für Medienhäuser und greift damit lange nicht so weit wie das EU-Recht.

Kompliziert werden internationale Kombinationen: eine Schweizer Internetfirma
hostet und entwirft den Auftritt für ein EU-Unternehmen und so weiter.
Neben diesen juristischen Finessen ist es bei eBusiness-Applikationen aus
Gründen der Transparenz und Offenheit wichtig, den Kunden nicht mit anonymen
Partnern arbeiten zu lassen. Ein Webauftritt, der den Betreiber im Dunkeln
lässt, wird kaum als seriös eingestuft und wird Mühe haben, zu den Kunden
eine bindende Beziehung herzustellen. Die meisten potentiellen Kunden werden
sich sträuben Informationen wie Adresse oder gar die Nummer der Kreditkarte
bekannt zu geben.

\subsubsection*{Disclaimer}

Die juristische Relevanz von Disclaimern wird in der Praxis zumeist überschätzt.
Grossfirmen haben gelegentlich am Ende ihrer Mails einen Hinweis auf die
Vertraulichkeit des Inhalts und dass bei fehlgeleiteten Mails deren Inhalt
nicht weitergegeben werden darf. Es ist halt ein Versuch, aber auch nicht
mehr. Denn juristisch entsprechen diese Hinweise Geschäftsbedingungen.
Und diese sind vor einer Geschäftshandlung bekannt zugeben und zu akzeptieren.
Korrekt, aber praktisch nicht machbar wäre es, zuerst den Disclaimer zu
senden und--falls der Empfänger die Bedingungen akzeptiert--erst anschliessend
den Inhalt.

Ebenso unwirksam sind die meisten Disclaimer auf Webseiten. Es ist--auch vor
Gericht--nicht glaubwürdig auf andere Webseiten zu verweisen und sich gleichzeitig
von diesen zu distanzieren. Wer in seriöser Absicht auf eine andere Webseite
verlinkt, sollte deren Inhalt auch kennen. Wenn der Inhalt nach dem Einbau des
Links verändert und eventuell juristisch bedenklich geworden ist, wird einem
das kaum als Vergehen angerechnet werden. Allerdings muss man nach einem
entsprechenden Hinweis darauf reagieren und seinen Eintrag anpassen.

\subsubsection*{Urheberrechtsschutz}

Ein juristisches Thema, das ebenfalls im Kapitel über eDistribution erwähnt
wird, ist das Urheberrecht. Es bezieht sich generell auf alle Medien. Hier
sei besonders auf die Situation im Bildbereich hingewiesen. Für die
Gestaltung der Produktkataloge werden zur Illustration oft Produktfotos
eingesetzt. Selbst wenn Sie diese nur von der Herstellerseite herunterladen
und verwenden, verstossen Sie gegen das Urheberrecht. Ausser die Fotos
sind explizit zur freien Verwendung deklariert. Selbst wer Fotos kauft,
muss sich über die Rechte seines Verkäufers informieren. Wenn später der
Fotograf seine Rechte geltend macht, kann die Verantwortung zwar an den
Verkäufer weitergegeben werden, nur hat man Pech, falls dieser Konkurs
gegangen ist oder nicht mehr auffindbar ist. Daher verwenden viele Firmen
für Ihre Webauftritte ausschliesslich Fotos, deren Urheberrecht sie selber
haben. Die also durch einen Fotografen speziell für sie erstellt werden.

% ===========================================================================
\section{Bestandteile des eBusiness}

\subsection{Übersicht der Wertschöpfungskette}

\subsection{Elemente der Wertschöpfungskette}

\subsubsection*{eProducts \& eServices}

Für den Anbieter von Waren ist die Präsentation seiner Produkte ein
zentraler Punkt. Die Darstellung, Beschreibung mit Attributen und
Fotos und Gliederung ermöglicht es den potentiellen Kunden ein
gewünschtes Produkt zu finden und bestenfalls auch zu bestellen. Für
diesen Zweck muss der Produktkatalog ständig auf dem aktuellen Stand
gehalten werden. Ein Aufwand, der bei grossen Systemen, die mehrere
hunderttausend Artikel enthalten können, nicht zu unterschätzen ist.
Bei komfortableren Systemen gibt es Möglichkeiten die Produkte hierarchisch
in Warengruppen zu gliedern. Damit können einige Produktattribute eventuell
schon auf der höheren Hierarchiestufe beschrieben werden und müssen
nicht redundant erstellt und gewartet werden.

Je nach Produkt besteht für den Kunden auch die Option seine Artikel
selber zu konfigurieren (zum Beispiel Dell: Konfiguration beziehungsweise
Modifikation von Ausstattungsvorschlägen). Dann benötigen einzelne
Komponenten noch weitere technische Angaben (Konfigurationsregeln), um
technisch falsche oder sinnlose Kombinationen zu verhindern. Der Aufbau
eines Konfigurationssystems ist natürlich aufwendiger und es hat auch
Auswirkungen auf die Produktion und Distribution der Produkte. Sie können
unter Umständen erst nach der Bestellung hergestellt werden, wenn es
sich bei der Konfiguration nicht nur um eine sehr geringe Anzahl an
Variationsmöglichkeiten handelt.

\subsubsection*{eProcurement}

Unter \emph{procurement} wird der Beschaffungsprozess in einem Unternehmen
verstanden. Mit dem E-Zusatz versehen, handelt es sich dann um die durch
spezielle Applikationen unterstützte elektronische Variante. Es
werden generell 3 Varianten unterschieden:

\begin{itemize}
    \item\textbf{Sell-Side}: Das System ist auf Verkäuferseite installiert.
    Der Aufwand mehrere Lieferantensysteme zu bedienen liegt beim Käufer.
    \item\textbf{Buy-Side}: Das System ist auf der Käuferseite installiert.
    Mit einem System können mehrere Lieferanten behandelt werden.
    \item\textbf{Marktplatz}: Plattform mit mehreren Anbietern und Nachfragern.
    Der Betreiber ist in der Regel eine unabhängige Drittfirma .
\end{itemize}

\subsubsection*{eMarketing}

Im Marketing gibt es die bekannte AIDA-Formel. Die Abkürzung steht für
\emph{Attention}, \emph{Interest}, \emph{Desire} und \emph{Action}.
So wie viele Erkenntnisse des klassischen Marketings ist auch diese
Formel im eMarketing gültig. Natürlich gibt es zusätzlich jede Menge
systemspezifische Merkmale. Die Einteilung der Kunden in Gruppen,
die dann vom Marketing separat betrachtet und mit Massnahmen gesegnet
werden können ist etwas anders als im klassischen Marketing. Wir
unterscheiden im eMarketing folgende Gruppen:

\begin{itemize}
    \item Online Surfer
    \item Online Consumer
    \item Online Prosumer \\
    (Mischung aus Produzent und Consumer--trägt zur Wertschöpfungskette
    positiv bei und ist gleichzeitig auch Konsument)
    \item Online Buyer und
    \item Online Key Customer
\end{itemize}


Ein bei Webshops häufig vernachlässigtes Gebiet ist die Verkaufs-Psychologie.
Wenn wir uns in Warenhäusern oder Supermärkten bewegen ist unsere gesamte
Umgebung nach psychologischen Kenntnissen gestaltet. Das ist nicht nur
die Akustik (also zumeist eine Hintergrundmusik), sondern kann auch
olfaktorisch (Geruchsstoffe) und optisch über die Beleuchtung erreicht
werden. Es ist zum Beispiel bekannt, dass die Gestaltung des Bodens
(hart oder weich) einen starken Einfluss auf unsere Gehgeschwindigkeit hat
und wird entsprechend eingesetzt. Eine unvollständige und etwas willkürliche
Aufzählung finden Sie zum Beispiel unter
\href{http://www.orbit9.de/wissen/verkaufspsychologie.php}
{http://www.orbit9.de/wissen/verkaufspsychologie.php}.
Mittlerweile gibt es aber auch im online-Bereich schon mehr Erfahrung,
die auch in professionellen Systemen eingesetzt wird. Diese Erkenntnisse
werden aber nicht sehr freizügig verbreitet. Wie sehr solche psychologischen
Tricks dann trotzdem auf Personen wirken, die sie kennen ist dann noche
eine weitere Diskussion. Das Wissen, dass zuviel Alkoholkonsum Kopfschmerzen
verursacht verhindert den Konsum ja auch nicht vollständig.

Weitere wichtige Punkte im eMarketing sind cross-selling und up-selling,
die noch im weiteren Verlauf des Skriptes behandelt werden.

\subsubsection*{eContracting}

\begin{itemize}
    \item\textbf{elektronischer Verhandlungsprozess} \\
    Softwaresysteme, die den elektronischen Verhandlungsprozess
    unterstützen, müssen die Dokumente und Unterlagen des gesamten
    Geschäftsprozesses verwalten und archivieren. Dazu gehören sämtliche
    Vereinbarungen und Vertragsabschlüsse, aber auch die digitalen
    Signaturen und das Controlling nach dem Vertragsabschluss.
    \item\textbf{digitale Signatur} \\
    Um rechtsgültige Vertragsabschlüsse im Internet zu erhalten,
    müssen sich die Geschäftspartner eindeutig identifizieren.
    Dies geschieht mittels digitaler Signaturen
    (siehe auch Kapitel~\ref{subsec:digitale-signatur} auf Seite~\ref{subsec:digitale-signatur}).
    Dokumente, die mit gültigen Signaturen versehen sind, sind rechtswirksam.
\end{itemize}

\subsubsection*{eDistribution}

Die Auswahl eines Distributionssystems sorgt für die Verteilung der Produkte,
Waren oder Dienstleistungen an die Kunden. Die Definition des Distributionskanals legt
fest, ob ein Produkt direkt oder indirekt abgesetzt wir. Beim indirekten Absatz
findet der Vertrieb nicht direkt zwischen dem Hersteller und dem Endkunden statt,
sondern es gibt eine oder mehrere Zwischenhändlerstufen. Bei digitalen Produkten
nennen sich die Zwischenhändler auch Infomediäre. Eine weitere strategische Aufgabe
ist die Bestimmung der Distributionslogistik. Hierbei wird definiert, ob eine Ware
gelagert wird oder in \emph{just-in-time} produziert wird, auf welchem Transportnetz und
mit welchem Service die Verteilung stattfindet.

\begin{itemize}
    \item\textbf{online-Distribution} \\
    Bei einer reinen online-Distribution findet kein materieller Güterfluss statt.
    Musikvermarkter sind ein gutes Beispiel für diese Kategorie. Bei der
    online-Distribution können aber genau wie beim materiellen Güterfluss
    direkte und indirekte Absatzkanäle gewählt werden. Auch bei open-source
    Sotwareprodukten werden häufig indirekte Kanäle in Form von Spiegelservern
    verwendet, die von anderen Organisationen als dem Hersteller betrieben werden.
    \item\textbf{offline-Distribution} \\
    Für diesen Begriff gibt es mehrere Definitionen. So wird der Begriff
    offline-Distribution verwendet, wenn in Abgrenzung zur online-Distribution,
    ein materieller Güterfluss vorliegt. Dies kann durchaus auch bei digitalen
    Gütern sinnvoll sein, um Medien mit grosser Qualität zu übertragen oder zum
    Schutz der Urheberschaft. Ausserdem redet man von offline-Distribution, wenn
    Medien innerhalb eines Intranets zur Verfügung stehen, ohne dass ein Zugriff
    auf das Internet erfolgt. Das kann bei grossen Datenmengen wie Filmen oder
    Datenbanken sinnvoll sein.
    \item\textbf{hybride Distribution} \\
    Bei der hybriden Distribution werden on- und offline-Distribution kombiniert.
    Wenn die Produkte aus materiellen Gütern bestehen, können Zusätze wie
    Handbücher, Software, Firmware und so weiter separat elektronisch verteilt
    werden. Der Materiallieferung müssen dann keine CDs, DVDs oder sonstigen
    Datenträger mehr beiliegen. Damit ist auch eine höhere Aktualität bei Updates
    gewährleistet.

    Ein grosses Thema bei digitalen Produkten ist der Urheberrechtsschutz. Ohne
    besondere Vorkehrungen hat der Hersteller keine Kontrolle über die weitere
    Herstellung und Verteilung von Kopien. Bei Fotos können digitale Wasserzeichen
    eingesetzt werden, um die Urheberschaft zu kennzeichnen. Bei Soft4 wareprodukten
    werden oft sogenannte Dongles eingesetzt, die auf klassische Weise an den
    Kunden verschickt werden. Oft kann die Software nach dem Download für eine
    begrenzte Zeit ohne Einschränkungen verwendet werden und erst nach Ablauf
    einer Frist wird die Verwendung eines Product-Keys oder Dongles überprüft.
\end{itemize}

\subsubsection*{ePayment}

\subsection{Usability}

\subsection{Personalisierung}

\subsection{Personalisierung}

\subsection{Passantenfunktion}

\subsection{Preisfindung}

\subsection{Auftragsbestätigung}

\subsection{Cross- und Up-Selling}

\subsection{Data-Mining}

\subsection{Warenkorb}

\subsection{Produkte Auswahl}

\subsection{Glaubwürdigkeit des eMarketing}

% ===========================================================================
\section{Informationssicherheit/Datensicherheit}

\subsection{Ziele der Informationssicherheit}

\subsection{Erwartungen an die Informationssicherheit}

\subsection{Angriffe}
\subsubsection*{Spionage}
\subsubsection*{Passwort-Cracker/Passwort-Guesser}
\subsubsection*{Horcher und \emph{The-Man-in-the-Middle}}
\subsubsection*{E-Shop Lifting}
\subsubsection*{Session Hijacking}
\subsubsection*{Viren, Würmer und anderes Getier}
\subsubsection*{DoS Attacken, Trojaner und Hintertüren}

\subsection{Autorisierung/Authentifizierung}

% ===========================================================================
\section{Kryptographie}

\subsection{Überblick}

\subsection{Geschichte}
\subsubsection*{Skytale}
\subsubsection*{Cäsar-Verfahren}
\subsubsection*{Monoalphabetische Substitutionen}

\subsection{Verschlüsselungsverfahren (symmetrisch, asymmetrisch und hybrid)}
\subsubsection*{Public Key}

\subsection{Das XOR-Verfahren}

\subsection{Digitale Signatur}
\label{subsec:digitale-signatur}

\subsection{Mathematische Grundlagen zu Krypto-Verfahren}
\subsubsection*{ggT}
\subsubsection*{Primzahlen}
\subsubsection*{Modulo p}
\subsubsection*{Inverses Modulo p}
\subsubsection*{Einfach \& Schwierig}

\subsection{Hashfunktionen}
\subsubsection*{Standard-Hashfunktionen}
\subsubsection*{salted Hash}

\subsection{Das Verfahren von Diffie/Hellmann}
\subsubsection*{Ausgangslage}
\subsubsection*{Vorgehen im Diffie/Hellman-Verfahren}
\subsubsection*{Ein Zahlenbeispiel}

\subsection{Das Verfahren von Rivest, Shamir und Adleman (RSA)}
\subsubsection*{Vorgehen im RSA-Verfahren}
\subsubsection*{Ein Zahlenbeispiel}

\subsection{Der eigene öffentliche Schlüssel}

\subsection{Java Hilfsprogramme}
\subsubsection*{\texttt{XorKryptRandom}}
\subsubsection*{Grösster gemeinsamer Teiler: \texttt{GCD}}
\subsubsection*{Das Inverse modulo einer Primzahl: \texttt{MInv}}
\subsubsection*{Potenzieren modulo einer Primzahl: \texttt{AhBmC}}

% ===========================================================================
\section{Implementierung}

\subsection{Anpassungen - Change Management}

\subsection{Session}

% ===========================================================================
\section{Übungen und Aufgaben}

\subsection{Warenkorb}
\subsubsection*{Alternative: Neuer Shop}

\subsection{Shop-Vergleich}

\subsection{Sicherheit}
\subsubsection*{Schutz gegen Angriffe}
\subsubsection*{Angriff}

\subsection{Verschlüsselung}

\subsection{Verschlüsselung - Praxis}

\subsection{Standard Web-Shops}

% ===========================================================================
\section{GnuPG}

\subsection{Installation}

\subsection{GPG-Home Verzeichnis}

\subsection{Schlüssel generieren}

\subsection{Importieren von Schlüsseln}

\subsection{Schlüssel unterschreiben und beglaubigen}

\subsection{Verschlüsseln / Entschlüsseln einer Botschaft}

