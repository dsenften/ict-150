\documentclass[12pt,a4paper]{exam}
\usepackage{natbib}

\printanswers % If you want to print answers
%\noprintanswers % If you don't want to print answers

\renewcommand{\solutiontitle}{\noindent\textbf{Lösung:}\enspace}

\RequirePackage{talent-factory}
\RequirePackage[top=3cm,bottom=2cm,left=2cm,right=1.5cm,headsep=10pt,a4paper]{geometry} % Page margins

\newcommand{\class}{Modul 150}
\newcommand{\examnum}{Test 1}
\newcommand{\examdate}{13.03.2019}
\newcommand{\timelimit}{45 Minuten}

\pagestyle{head}
\firstpageheader{}{}{}
\runningheader{\class}{\examnum\ - Seite \thepage\ / \numpages}{\examdate}
\runningheadrule

\begin{document}

    \setmainfont{Verdana}
    \parindent0em\parskip1em

    % Deutsche Schreibweise diverser Labels
    \pointpoints{Punkt}{Punkte}


    \noindent
    \begin{tabular*}{\textwidth}{l @{\extracolsep{\fill}} r @{\extracolsep{6pt}} l}
        \textbf{\class}           & \textbf{Name:} & \makebox[2in]{\hrulefill} \\
        \textbf{\examnum}         & & \\
        \textbf{\examdate}        & & \\
        \textbf{Zeit: \timelimit} & Dozent: & \makebox[2in]{Daniel Senften}
    \end{tabular*}\\
    \rule[1ex]{\textwidth}{1pt}

    Dieser Test besteht aus \numpages\ Seiten (inkl. Titelseite) und
    \numquestions\ Fragen. Wenn alle Fragen richtig beantwortet werden,
    dann sind maximal \numpoints\ möglich. Fragestellung jeweils gut lesen
    und bei den Antworten auf eine \emph{lesbare} Schrift achten.


    \begin{center}
        \textbf{Punktetabelle}\footnote{Diese Tabelle wird durch den Dozenten
        nach der Prüfung ausgefüllt.} \\ \vskip1em
        \addpoints

        \hqword{Aufgabe}
        \hpword{Punkte}
        \htword{\textbf{Total}}
        \hsword{Erreicht}
        \gradetable[h][questions]
    \end{center}

    \noindent
    \rule[1ex]{\textwidth}{1pt}

    % ===========================================================================

    \begin{questions}

        % -----------------------------------------------
        \question Geben Sie zu den folgenden möglichen Angriffe aus dem Kapitel
        Sicherheit eine mögliche Abhilfe.

        \begin{parts}
            \part[5] Passwort Cracker und Guesser
            \ifprintanswers
            \begin{solution}
                Verwenden von sicheren Passwörtern (keine bekannte Daten vie
                Geburtstag oder ähnliche Informationen). Verwenden von Ziffern,
                Buchstaben und Spezialzeichen. Es existieren auch etliche
                Passwort-Generatoren (Beispiel: \href{https://www.dashlane.com/}
                {Dashline}). Passwörter regelmässig ändern.
            \end{solution}
            \else\makeemptybox{2in}
            \fi

            \part[5] E-Shop Lifting
            \ifprintanswers
            \begin{solution}
                Keine sensiblen Daten auf dem Client ablegen (\emph{hidden fields},
                cockies$\ldots$). Falls Daten von einer Session zurr nächsten verwendet
                werden sollen, dann müssen diese zwingend auf dem Server (und nicht
                auf dem Client) abgelegt werden.
            \end{solution}
            \else\makeemptybox{2in}
            \fi

            \part[5] DoS Attacken
            \ifprintanswers
            \begin{solution}
                TCP/IP Protokollstack absichern. DDoS-Angriffe auf Anwendungsebene
                (wie HTTP-GET-/POST-Floods) werden direkt am Netzwerkrand bekämpft,
                damit sie den Anwendungsursprung nicht erreichen können und die
                Anwendungssicherheit gewährleistet bleibt. Zusätzlich können
                Angriffe auf die Anwendung mit Hilfe einer Web Application Firewall
                (WAF) verhindert.
            \end{solution}
            \else\makeemptybox{2in}
            \fi

            \part[5] Man-in-the-Middle
            \ifprintanswers
            \begin{solution}
                E-Mails verschlüsseln und signieren; Wenn nötig verschlüsseltes
                \textit{Surfen}. Generell sollte ausschliesslich eine SSL (\texttt{https})
                Verbindung verwendet werden. Unbekannte, öffentliche WLAN's vermeiden.
            \end{solution}
            \else\makeemptybox{2in}
            \fi

        \end{parts}

        \addpoints


        % -----------------------------------------------
        \question Beschreiben Sie, wie Sie als \textit{Hacker} vorgehen würden,
        um Sicherheitslöcher im eBanking auszunutzen. Was ist zu tun, um in
        das System einzudringen, dieses auszuhorchen oder zu manipulieren.


        \begin{parts}
            \part[5] User achtet nicht auf sichere Verbindung.
            \ifprintanswers
            \begin{solution}
                Net-Sniffer (z.B.~\href{https://www.wireshark.org}{Wireshark})
                einsetzen und Transfer analysieren. Alle besuchten Webseiten mit
                Passworteingaben notieren. Alle persönlichen Informationen des
                Users sicherstellen.
            \end{solution}
            \else\makeemptybox{2in}
            \fi

            \part[5] User gibt telefonisch Passwörter durch.
            \ifprintanswers
            \begin{solution}
                Person unter falschem Namen (Bank) anrufen; Telefongespräche abhören.
            \end{solution}
            \else\makeemptybox{2in}
            \fi

            \part[5] User beantwortet alle (auch \emph{Fake-}) E-Mails.
            \ifprintanswers
            \begin{solution}
                \emph{Fake-} E-Mails verfassen (Stichwort:
                \href{https://de.wikipedia.org/wiki/Phishing}{Phishing}).
            \end{solution}
            \else\makeemptybox{2in}
            \fi

            \part[5] Benutzer verwendet keinen Virenschutz.
            \ifprintanswers
            \begin{solution}
                Trojaner via Virus einschleusen.
            \end{solution}
            \else\makeemptybox{2in}
            \fi

            \part[5] Benutzer hat den Cache nicht geleert.
            \ifprintanswers
            \begin{solution}
                Trojaner nach Browser-Cache-Dateien suchen lassen und anschliessend
                übermitteln.
            \end{solution}
            \else\makeemptybox{2in}
            \fi

        \end{parts}

        \addpoints

        % -----------------------------------------------
        \newpage
        \question[10] Als Webentwickler sollte man die wichtigsten Begriffe und
        Gepflogenheiten aus dem Bereich der `\emph{Juristischen Grundlagen}'
        kennen.

        Nenne drei Begriffe und erkläre diese kurz.

        \ifprintanswers
        \begin{solution}
            Für jeden Begriff wird je ein Punkt gutgeschrieben und jede
            korrekte Erklärung wird mit zwei Punkten bewertet.
            \begin{itemize}
                \item Abmahnung
                \item Impressum
                \item Disclaimer | Allgemeine Geschäftsbedingungen
                \item Urheberrechtsschutz
            \end{itemize}
        \end{solution}
        \else\makeemptybox{\fill}
        \fi

        \addpoints

        % -----------------------------------------------
        \newpage
        \question[15] Was verstehen wir unter der Wertschöpfungskette
        (Bestandteile des \emph{E-Business}).

        Nenne fünf dieser Begriffe und erkläre diese kurz.

        \ifprintanswers
        \begin{solution}
            Für jeden Begriff wird je ein Punkt gutgeschrieben und jede
            korrekte Erklärung wird mit zwei Punkten bewertet.
            \begin{itemize}
                \item eProduct | eService
                \item eProcurement
                \item eMarketing
                \item eContracting
                \item eDistribution
                \item ePayment
            \end{itemize}
        \end{solution}
        \else\makeemptybox{\fill}
        \fi

        \addpoints

        % -----------------------------------------------
        \newpage
        \question[10] Eine gute Web-Applikation sollte die Vielfalt der
        Anforderungen abdecken und ergonomisch gestaltet sein. Nenne und
        erläutere Möglichkeiten, die Benutzerfreundlichkeit
        (\emph{Usability}) zu verbessern.

        \ifprintanswers
        \begin{solution}
            Für jedes Stichwort (max. 5) gibt es je einen Punkt und je einen
            Punkt für eine korrekte Erläuterung des erwähnten Begriffes.
            \begin{itemize}
                \item Performance
                \item Verfügbarkeit
                \item Plattformunabhängigkeit
                \item Barrierefreiheit
                \item Lesbarkeit
                \item Seitenlänge
                \item Orientierung
                \item Konsistenz
            \end{itemize}
        \end{solution}
        \else\makeemptybox{\fill}
        \fi

        \addpoints

    \end{questions}

\end{document}
